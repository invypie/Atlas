\chapter{Survey of the Elementary Principles}

The motion of material bodies formed the subject of some of the earliest research pursued by the pioneers of physics. From their efforts there has evolved a vast field known as analytical mechanics or dynamics, or simply, mechanics. In the present century the term ``classical mechanics" has come into wide use to denote this branch of physics in contradistinction to the newer physical theories, especially quantum mechanics. We shall follow this usage, interpreting the name to include the type of mechanics arising out of the special theory of relativity. It is the purpose of this book to develop the structure of classical mechanics and to outline some of its applications of present-day interest in pure physics. Basic to any presentation of mechanics are a number of fundamental physical concepts, such as space, time, simultaneity, mass, and force. For the most part, however, these concepts will not be analyzed critically here; rather, they will be assumed as undefined terms whose meanings are familiar to the reader.

\section{Definition and Existence of the Integral}

\begin{definition}\label{definition:6.1}
    Let \(\left[a,b\right]\) be a given interval. By a \emph{partition} \(P\) of \(\left[a,b\right]\) we mean a finite set of points \(x_0\), \(x_1\), \(\ldots\), \(x_n\), where
    \begin{equation*}
        a=x_0\leqslant x_1\leqslant\cdots\leqslant x_{n-1}\leqslant x_n=b.
    \end{equation*}
    We write
    \begin{equation*}
        \increment x_i=x_i-x_{i-1}\qquad\text{(\(i=1,\ldots,n\))}.
    \end{equation*}
    Now suppose \(f\) is a bounded real function defined on \(\left[a,b\right]\). Corresponding to each partition \(P\) of \(\left[a,b\right]\) we put
    \begin{align*}
        M_i&=\sup f\left(x\right)\qquad\text{(\(x_{i-1}\leqslant x\leqslant x_i\))},\\
        m_i&=\inf f\left(x\right)\qquad\text{(\(x_{i-1}\leqslant x\leqslant x_i\))},\\
        U\left(P,f\right)&=\sum_{i=1}^nM_i\increment x_i,\\
        L\left(P,f\right)&=\sum_{i=1}^nm_i\increment x_i,
    \end{align*}
    and finally
    \begin{align}
        \upint_a^bf\odif{x}&=\inf U\left(P,f\right),\label{eq:6.1}\\
        \lowint_a^bf\odif{x}&=\sup L\left(P,f\right),\label{eq:6.2}
    \end{align}
    where the \(\inf\) and the \(\sup\) are taken over all partitions \(P\) of \(\left[a,b\right]\). The left members of \eqref{eq:6.1} and \eqref{eq:6.2} are called the \emph{upper} and \emph{lower Riemann integrals} of \(f\) over \(\left[a,b\right]\), respectively.

    If the upper and lower integrals are equal, we say that \(f\) is \emph{Riemann-integrable} on \(\left[a,b\right]\), we write \(f\in\symscr{R}\) (that is, \(\symscr{R}\) denotes the set of Riemann-integrable functions), and we denote the common value of \eqref{eq:6.1} and \eqref{eq:6.2} by
    \begin{equation}
        \int_a^bf\odif{x}
    \end{equation}
    or by
    \begin{equation}
        \int_a^bf\left(x\right)\odif{x}.
    \end{equation}

    This is the \emph{Riemann integral} of \(f\) over \(\left[a,b\right]\). Since \(f\) is bounded, there exist two numbers, \(m\) and \(M\), such that
    \begin{equation*}
        m\leqslant f\left(x\right)\leqslant M\qquad\text{(\(a\leqslant x\leqslant b\))}.
    \end{equation*}
    Hence, for every \(P\)
    \begin{equation*}
        m\left(b-a\right)\leqslant L\left(P,f\right)\leqslant U\left(P,f\right)\leqslant M\left(b-a\right),
    \end{equation*}
    so that the numbers \(L\left(P,f\right)\) and \(U\left(P,f\right)\) form a bounded set. This shows that \emph{the upper and lower integrals are defined for every} bounded function \(f\). The question of their equality, and hence the question of the integrability of \(f\), is a more delicate one. Instead of investigating it separately for the Riemann integral, we shall immediately consider a more general situation.
\end{definition}

\begin{definition}
    Let \(\alpha\) be a monotonically increasing function on \(\left[a,b\right]\) (since \(\alpha\left(a\right)\) and \(\alpha\left(b\right)\) are finite, it follows that \(\alpha\) is bounded on \(\left[a,b\right]\)). Corresponding to each partition \(P\) of \(\left[a,b\right]\), we write
    \begin{equation*}
        \increment\alpha_i=\alpha\left(x_i\right)-\alpha\left(x_{i-1}\right).
    \end{equation*}
    It is clear that \(\increment\alpha_i\geqslant0\). For any real function \(f\) which is bounded on \(\left[a,b\right]\) we put
    \begin{align*}
        U\left(P,f,\alpha\right)=\sum_{i=1}^nM_i\increment\alpha_i,\\
        L\left(P,f,\alpha\right)=\sum_{i=1}^nm_i\increment\alpha_i,
    \end{align*}
    where \(M_i\), \(m_i\) have the same meaning as in Definition~\ref{definition:6.1}, and we define
    \begin{align}
        \upint_a^bf\odif{\alpha}=\inf U\left(P,f,\alpha\right),\label{eq:6.5}\\
        \lowint_a^bf\odif{\alpha}=\sup L\left(P,f,\alpha\right),\label{eq:6.6}
    \end{align}
    the \(\inf\) and \(\sup\) again being taken over all partitions.

    If the left members of \eqref{eq:6.5} and \eqref{eq:6.6} are equal, we denote their common value by
    \begin{equation}
        \int_a^bf\odif{\alpha}\label{eq:6.7}
    \end{equation}
    or sometimes by
    \begin{equation}
        \int_a^bf\left(x\right)\odif{\alpha\left(x\right)}.\label{eq:6.8}
    \end{equation}

    This is the \emph{Riemann--Stieltjes integral} (or simply the \emph{Stieltjes integral}) of \(f\) with respect to \(\alpha\), over \(\left[a,b\right]\).

    If \eqref{eq:6.7} exists, i.e., if \eqref{eq:6.5} and \eqref{eq:6.6} are equal, we say that \(f\) is integrable with respect to \(\alpha\), in the Riemann sense, and write \(f\in\symscr{R}\left(\alpha\right)\).
\end{definition}

By taking \(\alpha\left(x\right)=x\), the Riemann integral is seen to be a special case of the Riemann--Stieltjes integral. Let us mention explicitly, however, that in the general case \(\alpha\) need not even be continuous.

A few words should be said about the notation. We prefer \eqref{eq:6.7} to \eqref{eq:6.8}, since the letter \(x\) which appears in \eqref{eq:6.8} adds nothing to the content of \eqref{eq:6.7}. It is immaterial which letter we use to represent the so-called ``variable of integration." For instance, \eqref{eq:6.8} is the same as
\begin{equation*}
    \int_a^bf\left(y\right)\odif{\alpha\left(y\right)}.
\end{equation*}
The integral depends on \(f\), \(\alpha\), \(a\) and \(b\), but not on the variable of integration, which may as well be omitted.

The role played by the variable of integration is quite analogous to that of the index of summation: The two symbols
\begin{equation*}
    \sum_{i=1}^nc_i,\qquad\sum_{k=1}^nc_k
\end{equation*}
are the same, since each means \(c_1+c_2+\cdots+c_n\).

Of course, no harm is done by inserting the variable of integration, and in many cases it is actually convenient to do so.

We shall now investigate the existence of the integral \eqref{eq:6.7}. Without saying so every time, \(f\) will be assumed real and bounded, and \(\alpha\) monotonically increasing on \(\left[a,b\right]\); and, when there can be no misunderstanding, we shall write \(\int\) in place of \(\int_a^b\).

\begin{definition}
    We say that the partition \(P^*\) is a \emph{refinement} of \(P\) if \(P^*\supset P\) (that is, if every point of \(P\) is a point of \(P^*\)). Given two partitions, \(P_1\) and \(P_2\), we say that \(P^*\) is their \emph{common refinement} if \(P^*=P_1\cup P_2\).
\end{definition}

\begin{theorem}\label{theorem:6.1}
    If \(P^*\) is a refinement of \(P\), then
    \begin{equation}
        L\left(P,f,\alpha\right)\leqslant L\left(P^*,f,\alpha\right)\label{eq:6.9}
    \end{equation}
    and
    \begin{equation}
        U\left(P^*,f,\alpha\right)\leqslant U\left(P,f,\alpha\right).\label{eq:6.10}
    \end{equation}
\end{theorem}

\begin{proof}
    To prove \eqref{eq:6.9}, suppose first that \(P^*\) contains just one point more than \(P\). Let this extra point be \(x^*\), and suppose \(x_{i-1}<x^*<x_i\), where \(x_{i-1}\) and \(x_i\) are two consecutive points of \(P\). Put
    \begin{align*}
        w_1&=\inf f\left(x\right)\qquad\text{(\(x_{i-1}\leqslant x\leqslant x^*\))},\\
        w_2&=\inf f\left(x\right)\qquad\text{(\(x^*\leqslant x\leqslant x_i\))}.
    \end{align*}
    Clearly \(w_1>m_i\) and \(w_2>m_i\), where, as before,
    \begin{equation*}
        m_i=\inf f\left(x\right)\qquad\text{(\(x_{i-1}\leqslant x\leqslant x_i\))}.
    \end{equation*}
    Hence
    \begin{align*}
        &\phantom{{}={}}L\left(P^*,f,\alpha\right)-L\left(P,f,\alpha\right)\\
        &=w_1\left[\alpha\left(x^*\right)-\alpha\left(x_{i-1}\right)\right]+w_2\left[\alpha\left(x_i\right)-\alpha\left(x^*\right)\right]-m_i\left[\alpha\left(x_i\right)-\alpha\left(x_{i-1}\right)\right]\\
        &=\left(w_1-m_i\right)\left[\alpha\left(x^*\right)-\alpha\left(x_{i-1}\right)\right]+\left(w_2-m_i\right)\left[\alpha\left(x_i\right)-\alpha\left(x^*\right)\right]\geqslant0.
    \end{align*}
    If \(P^*\) contains \(k\) points more than \(P\), we repeat this reasoning \(k\) times, and arrive at \eqref{eq:6.9}. The proof of \eqref{eq:6.10} is analogous.
\end{proof}

\begin{theorem}
    \(\lowint_a^bf\odif{\alpha}\leqslant\upint_a^bf\odif{\alpha}\).
\end{theorem}

\begin{proof}
    Let \(P^*\) be the common refinement of two partitions \(P_1\), and \(P_2\). By Theorem~\ref{theorem:6.1},
    \begin{equation*}
        L\left(P_1,f,\alpha\right)\leqslant L\left(P^*,f,\alpha\right)\leqslant U\left(P^*,f,\alpha\right)\leqslant U\left(P_2,f,\alpha\right)
    \end{equation*}
    Hence
    \begin{equation}
        L\left(P_1,f,\alpha\right)\leqslant U\left(P_2,f,\alpha\right).\label{eq:6.11}
    \end{equation}
    If \(P_2\) is fixed and the \(\sup\) is taken over all \(P_1\), \eqref{eq:6.11} gives
    \begin{equation}
        \lowint f\odif{\alpha}\leqslant U\left(P_2,f,\alpha\right).\label{eq:6.12}
    \end{equation}
    The theorem follows by taking the \(\inf\) over all \(P_2\) in \eqref{eq:6.12}.
\end{proof}

\begin{theorem}\label{theorem:6.3}
    \(f\in\symscr{R}\left(a\right)\) on \(\left[a,b\right]\) if and only if for every \(\varepsilon>0\) there exists a partition \(P\) such that
    \begin{equation}
        U\left(P,f,\alpha\right)-L\left(P,f,\alpha\right)<\varepsilon.\label{eq:6.13}
    \end{equation}
\end{theorem}

\begin{proof}
    For every \(P\) we have
    \begin{equation*}
        L\left(P,f,\alpha\right)\leqslant\lowint f\odif{\alpha}\leqslant\upint f\odif{\alpha}\leqslant U\left(P,f,\alpha\right).
    \end{equation*}
    Thus \eqref{eq:6.13} implies
    \begin{equation*}
        0\leqslant\upint f\odif{\alpha}-\lowint f\odif{\alpha}<\varepsilon.
    \end{equation*}
    Hence, if \eqref{eq:6.13} can be satisfied for every \(\varepsilon>0\), we have
    \begin{equation*}
        \upint f\odif{\alpha}=\lowint f\odif{\alpha},
    \end{equation*}
    that is, \(f\in\symscr{R}\left(\alpha\right)\).

    Conversely, suppose \(f\in\symscr{R}\left(\alpha\right)\), and let \(\varepsilon>0\) be given. Then there exist partitions \(P_1\) and \(P_2\) such that
    \begin{align}
        U\left(P_2,f,\alpha\right)-\int f\odif{\alpha}&<\frac{\varepsilon}{2},\label{eq:6.14}\\
        \int f\odif{\alpha}-L\left(P_1,f,\alpha\right)&<\frac{\varepsilon}{2}.\label{eq:6.15}
    \end{align}
    We choose \(P\) to be the common refinement of \(P_1\) and \(P_2\). Then Theorem~\ref{theorem:6.1}, together with \eqref{eq:6.14} and \eqref{eq:6.15}, shows that
    \begin{equation*}
        U\left(P,f,\alpha\right)\leqslant U\left(P_2,f,\alpha\right)<\int f\odif{\alpha}+\frac{\varepsilon}{2}<L\left(P_1,f,\alpha\right)+\varepsilon\leqslant L\left(P,f,\alpha\right)+\varepsilon,
    \end{equation*}
    so that \eqref{eq:6.13} holds for this partition \(P\).
\end{proof}

Theorem~\ref{theorem:6.3} furnishes a convenient criterion for integrability. Before we apply it, we state some closely related facts.

\begin{theorem}
    \leavevmode
    \begin{enumerate}
        \item\label{itm:6.1.1} If \eqref{eq:6.13} holds for some \(P\) and some \(\varepsilon\), then \eqref{eq:6.13} holds (with the same \(\varepsilon\)) for every refinement of \(P\).
        \item\label{itm:6.1.2} If \eqref{eq:6.13} holds for \(P=\set{x_0,\ldots,x_n}\) and if \(s_i\), \(t_i\) are arbitrary points in \(\left[x_{i-1},x_i\right]\), then
        \begin{equation*}
            \sum_{i=1}^n\abs[f\left(s_i\right)-f\left(t_i\right)]\increment\alpha_i<\varepsilon.
        \end{equation*}
        \item\label{itm:6.1.3} If \(f\in\symscr{R}\left(\alpha\right)\) and the hypotheses of \ref{itm:6.1.2} hold, then
        \begin{equation*}
            \abs[\sum_{i=1}^nf\left(t_i\right)\increment\alpha_i-\int_a^bf\odif{\alpha}]<\varepsilon.
        \end{equation*}
    \end{enumerate}
\end{theorem}

\begin{proof}
    Theorem~\ref{theorem:6.1} implies \ref{itm:6.1.1}. Under the assumptions made in \ref{itm:6.1.2}, both \(f\left(s_i\right)\) and \(f\left(t_i\right)\) lie in \(\left[m_i,M_i\right]\), so that \(\abs[f\left(s_i\right)-f\left(t_i\right)]\leqslant M_i-m_i\). Thus
    \begin{equation*}
        \sum_{i=1}^n\abs[f\left(s_i\right)-f\left(t_i\right)]\increment\alpha_i\leqslant U\left(P,f,\alpha\right)-L\left(P,f,\alpha\right),
    \end{equation*}
    which proves \ref{itm:6.1.2}. The obvious inequalities
    \begin{equation*}
        L\left(P,f,\alpha\right)\leqslant\sum f\left(t_i\right)\increment\alpha_i\leqslant U\left(P,f,\alpha\right)
    \end{equation*}
    and
    \begin{equation*}
        L\left(P,f,\alpha\right)\leqslant\int f\odif{\alpha}\leqslant U\left(P,f,\alpha\right)
    \end{equation*}
    prove \ref{itm:6.1.3}.
\end{proof}

\begin{theorem}\label{theorem:6.5}
    If \(f\) is continuous on \(\left[a,b\right]\) then \(f\in\symscr{R}\left(\alpha\right)\) on \(\left[a,b\right]\).
\end{theorem}

\begin{proof}
    Let \(\varepsilon>0\) be given. Choose \(\eta>0\) so that
    \begin{equation*}
        \left[\alpha\left(b\right)-\alpha\left(a\right)\right]\eta<\varepsilon.
    \end{equation*}
    Since \(f\) is uniformly continuous on \(\left[a,b\right]\) (Theorem~\ref{theorem:4.?}), there exists a \(\delta>0\) such that
    \begin{equation}
        \abs[f\left(x\right)-f\left(t\right)]<\eta\label{eq:6.16}
    \end{equation}
    if \(x\in\left[a,b\right]\), \(t\in\left[a,b\right]\), and \(\abs[x-t]<\delta\).

    If \(P\) is any partition of \(\left[a,b\right]\) such that \(\increment x_i<\delta\) for all \(i\),  then \eqref{eq:6.16} implies that
    \begin{equation}
        M_i-m_i\leqslant\eta\qquad\text{(\(i=1,\ldots,n\))}
    \end{equation}
    and therefore
    \begin{align*}
        U\left(P,f,\alpha\right)-L\left(P,f,\alpha\right)&=\sum_{i=1}^n\left(M_i-m_i\right)\increment\alpha_i\\
        &\leqslant\eta\sum_{i=1}^n\increment\alpha_i=\eta\left[\alpha\left(b\right)-\alpha\left(a\right)\right]<\varepsilon.
    \end{align*}
    By Theorem~\ref{theorem:6.3}, \(f\in\symscr{R}\left(\alpha\right)\).
\end{proof}

\begin{theorem}
    If \(f\) is monotonic on \(\left[a,b\right]\), and if \(\alpha\) is continuous on \(\left[a,b\right]\), then \(f\in\symscr{R}\left(\alpha\right)\). (We still assume, of course, that \(\alpha\) is monotonic.)
\end{theorem}

\begin{proof}
    Let \(\varepsilon>0\) be given. For any positive integer \(n\), choose a partition such that
    \begin{equation*}
        \increment\alpha_i=\frac{\alpha\left(b\right)-\alpha\left(a\right)}{n}\qquad\text{(\(i=1,\ldots,n\))}.
    \end{equation*}
    This is possible since \(\alpha\) is continuous (Theorem~\ref{theorem:4.?}).

    We suppose that \(f\) is monotonically increasing (the proof is analogous in the other case). Then
    \begin{equation*}
        M_i=f\left(x_i\right),\qquad m_i=f\left(x_{i-1}\right)\qquad\text{(\(i=1,\ldots,n\))},
    \end{equation*}
    so that
    \begin{align*}
        U\left(P,f,\alpha\right)-L\left(P,f,\alpha\right)&=\frac{\alpha\left(b\right)-\alpha\left(a\right)}{n}\sum_{i=1}^n\left[f\left(x_i\right)-f\left(x_{i-1}\right)\right]\\
        &=\frac{\alpha\left(b\right)-\alpha\left(a\right)}{n}\cdot\left[f\left(b\right)-f\left(a\right)\right]<\varepsilon
    \end{align*}
    if \(n\) is taken large enough. By Theorem~\ref{theorem:6.3}, \(f\in\symscr{R}\left(\alpha\right)\).
\end{proof}

\begin{theorem}
    Suppose \(f\) is bounded on \(\left[a,b\right]\), \(f\) has only finitely many points of discontinuity on \(\left[a,b\right]\), and \(\alpha\) is continuous at every point at which \(f\) is discontinuous. Then \(F\in\symscr{R}\left(\alpha\right)\).
\end{theorem}

\begin{proof}
    Let \(\varepsilon>0\) be given. Put \(M=\sup\abs[f\left(x\right)]\), let \(E\) be the set of points at which \(f\) is discontinuous. Since \(E\) is finite and \(\alpha\) is continuous at every point of \(E\), we can cover \(E\) by finitely many disjoint intervals \(\left[u_j,v_j\right]\subset\left[a,b\right]\) such that the sum of the corresponding differences \(\alpha\left(v_j\right)-\alpha\left(u_j\right)\) is less than \(\varepsilon\). Furthermore, we can place these intervals in such a way that every point of \(E\cap\left(a,b\right)\) lies in the interior of some \(\left[u_j,v_j\right]\).

    Remove the segments \(\left(u_j,v_j\right)\) from \(\left[a,b\right]\). The remaining set \(K\) is compact. Hence \(f\) is uniformly continuous on \(K\), and there exists \(\delta>0\) such that \(\abs[f\left(s\right)-f\left(t\right)]<\varepsilon\) if \(s\in K\), \(t\in K\), \(\abs[s-t]<\delta\).

    Now form a partition \(P=\set{x_0,x_1,\ldots,x_n}\) of \(\left[a,b\right]\), as follows: Each \(u_j\) occurs in \(P\). Each \(v_j\) occurs in \(P\). No point of any segment \(\left(u_j,v_j\right)\) occurs in \(P\). If \(x_{i-1}\) is not one of the \(u_j\), then \(\increment x_i<\delta\).

    Note that \(M_i-m_i\leqslant2M\) for every \(i\), and that \(M_i-m_i\leqslant\varepsilon\) unless \(x_{i-1}\) is one of the \(u_j\). Hence, as in the proof of Theorem~\ref{theorem:6.5},
    \begin{equation*}
        U\left(P,f,\alpha\right)-L\left(P,f,\alpha\right)\leqslant\left[\alpha\left(b\right)-\alpha\left(a\right)\right]\varepsilon+2M\varepsilon.
    \end{equation*}
    Since \(\varepsilon\) is arbitrary, Theorem~\ref{theorem:6.3} shows that \(f\in\symscr{R}\left(\alpha\right)\).
\end{proof}

\begin{note*}
    If \(f\) and \(\alpha\) have a common point of discontinuity, then \(f\) need not be in \(\symscr{R}\left(\alpha\right)\). Exercise~\ref{exercise:6.3} shows this.
\end{note*}

\begin{theorem}
    Suppose \(f\in\symscr{R}\left(\alpha\right)\) on \(\left[a,b\right]\), \(m\leqslant f\leqslant M\), \(\phi\) is continuous on \(\left[m,M\right]\), and \(h\left(x\right)=\phi\left(f\left(x\right)\right)\) on \(\left[a,b\right]\). Then \(h\in\symscr{R}\left(\alpha\right)\) on \(\left[a,b\right]\).
\end{theorem}

\begin{proof}
    Choose \(\varepsilon>0\). Since \(\phi\) is uniformly continuous on \(\left[m,M\right]\), there exists \(\delta>0\) such that \(\delta<\varepsilon\) and \(\abs[\phi\left(s\right)-\phi\left(t\right)]<\varepsilon\) if \(\abs[s-t]\leqslant\delta\) and \(s,t\in\left[m,M\right]\).

    Since \(f\in\symscr{R}\left(\alpha\right)\), there is a partition \(P=\set{x_0,x_1,\ldots,x_n}\) of \(\left[a,b\right]\) such that
    \begin{equation}
        U\left(P,f,\alpha\right)-L\left(P,f,\alpha\right)<\delta^2.\label{eq:6.18}
    \end{equation}
    Let \(M_i\), \(m_i\) have the same meaning as in Definition~\ref{definition:6.1}, and let \(M_i^*\), \(m_i^*\) be the analogous numbers for \(h\). Divide the numbers \(1\), \(\ldots\), \(n\) into two classes: \(i\in A\) if \(M_i-m_i<\delta\), \(i\in B\) if \(M_i-m_i\geqslant\delta\).

    For \(i\in A\), our choice of \(\delta\) shows that \(M_i^*-m_i^*\leqslant\varepsilon\).
    
    For \(i\in B\), \(M_i^*-m_i^*\leqslant2K\), where \(K=\sup\abs[\phi\left(t\right)]\), \(m\leqslant t\leqslant M\). By \eqref{eq:6.18}, we have
    \begin{equation}
        \delta\sum_{i\in B}\increment\alpha_i\leqslant\sum_{i\in B}\left(M_i-m_i\right)\increment\alpha_i<\delta^2
    \end{equation}
    so that \(\sum_{i\in B}\increment\alpha_i<\delta\). It follows that
    \begin{align*}
        U\left(P,h,\alpha\right)-L\left(P,h,\alpha\right)&=\sum_{i\in A}\left(M_i^*-m_i^*\right)\increment\alpha_i+\sum_{i\in B}\left(M_i^*-m_i^*\right)\increment\alpha_i\\
        &\leqslant\varepsilon\left[\alpha\left(b\right)-\alpha\left(a\right)\right]+2K\delta<\varepsilon\left[\alpha\left(b\right)-\alpha\left(a\right)+2K\right].
    \end{align*}
    Since \(\varepsilon\) was arbitrary, Theorem~\ref{theorem:6.3} implies that \(h\in\symscr{R}\left(\alpha\right)\).
\end{proof}

\begin{remark*}
    This theorem suggests the question: Just what functions are Riemann-integrable? The answer is given by Theorem~\ref{theorem:11.?}\ref{itm:11.?}.
\end{remark*}


\section{Characteristic Values}

The introductory remarks of the previous section provide us with a starting point for our attempt to analyze the general linear operator \(T\). We take our cue from \eqref{eq:6.2}, which suggests that we should study vectors which are sent by \(T\) into scalar multiples of themselves.

\begin{definition}
    Let \(V\) be a vector space over the field \(F\) and let \(T\) be a linear operator on \(V\). A \emph{characteristic value} of \(T\) is a scalar \(c\) in \(F\) such that there is a non-zero vector \(\alpha\) in \(V\) with \(T\alpha=c\alpha\). If \(c\) is a characteristic value of \(T\), then
    \begin{enumerate}
        \item any \(\alpha\) such that \(T\alpha=c\alpha\) is called a \emph{characteristic vector} of \(T\) associated with the characteristic value \(c\);
        \item the collection of all \(\alpha\) such that \(T\alpha=c\alpha\) is called the \emph{characteristic space} associated with \(c\).
    \end{enumerate}
\end{definition}

Characteristic values are often called characteristic roots, latent roots, eigenvalues, proper values, or spectral values. In this book we shall use only the name `characteristic values.'

If \(T\) is any linear operator and \(c\) is any scalar, the set of vectors \(\alpha\) such that \(T\alpha=c\alpha\) is a subspace of \(V\). It is the null space of the linear transformation \(\left(T-cI\right)\). We call \(c\) a characteristic value of \(T\) if this subspace is different from the zero subspace, i.e., if \(\left(T-cI\right)\) fails to be \(1\):\(1\). If the underlying space \(V\) is finite-dimensional, \(\left(T-cI\right)\) fails to be \(1\):\(1\) precisely when its determinant is different from \(0\). Let us summarize.

\begin{theorem}
    Let \(T\) be a linear operator on a finite-dimensional space \(V\) and let \(c\) he a scalar. The following are equivalent.
    \begin{enumerate}
        \item \(c\) is a characteristic value of \(T\).
        \item The operator \(\left(T-cI\right)\) is singular (not invertible).
        \item\label{itm:6.2.5} \(\det\left(T-cI\right)=0\).
    \end{enumerate}
\end{theorem}

The determinant criterion~\ref{itm:6.2.5} is very important because it tells us where to look for the characteristic values of \(T\). Since \(\det\left(T-cI\right)\) is a polynomial of degree \(n\) in the variable \(c\), we will find the characteristic values as the roots of that polynomial. Let us explain carefully.

If \(\symcal{B}\) is any ordered basis for \(V\) and \(A=\left[T\right]_{\symcal{B}}\), then \(\left(T-cI\right)\) is invertible if and only if the matrix \(\left(A-cI\right)\) is invertible. Accordingly, we make the following definition.

\begin{definition}
    If \(A\) is an \(n\times n\) matrix over the field \(F\), a \emph{characteristic value of \(A\) in \(F\)} is a scalar \(c\) in \(F\) such that the matrix \(\left(A-cI\right)\) is singular (not invertible).
\end{definition}

Since \(c\) is a characteristic value of \(A\) if and only if \(\det\left(A-cI\right)=0\), or equivalently if and only if \(\det\left(cI-A\right)=0\), we form the matrix \(\left(xI-A\right)\) with polynomial entries, and consider the polynomial \(f=\det\left(xI-A\right)\). Clearly the characteristic values of \(A\) in \(F\) are just the scalars \(c\) in \(F\) such that \(f\left(c\right)=0\). For this reason \(f\) is called the \emph{characteristic polynomial} of \(A\). It is important to note that \(f\) is a monic polynomial which has degree exactly \(n\). This is easily seen from the formula for the determinant of a matrix in terms of its entries.

\begin{lemma}
    Similar matrices have the same characteristic polynomial.
\end{lemma}

\begin{proof}
    It \(B=P^{-1}AP\), then
    \begin{align*}
        \det\left(xI-B\right)&=\det\left(xI-P^{-1}AP\right)\\
        &=\det\left(P^{-1}\left(xI-A\right)P\right)\\
        &=\det P^{-1}\cdot\det\left(xI-A\right)\cdot\det P\\
        &=\det\left(xI-A\right).\qedhere
    \end{align*}
\end{proof}

This lemma enables us to define sensibly the characteristic polynomial of the operator \(T\) as the characteristic polynomial of any \(n\times n\) matrix which represents \(T\) in some ordered basis for \(V\). Just as for matrices, the characteristic values of \(T\) will be the roots of the characteristic polynomial for \(T\). In particular, this shows us that \(T\) cannot have more than \(n\) distinct characteristic values. It is important to point out that \(T\) may not have any characteristic values.

\begin{example}\label{example:6.1}
    Let \(T\) be the linear operator on \(R^2\) which is represented in the standard ordered basis by the matrix
    \begin{equation*}
        A=
        \begin{bmatrix}
            0 & -1 \\
            1 & 0
        \end{bmatrix}
        .
    \end{equation*}
    The characteristic polynomial for \(T\) (or for \(A\)) is
    \begin{equation*}
        \det\left(xI-A\right)=
        \begin{vmatrix}
            x & 1 \\
            -1 & x
        \end{vmatrix}
        =x^2+1.
    \end{equation*}
    Since this polynomial has no real roots, \(T\) has no characteristic values. If \(U\) is the linear operator on \(C^2\) which is represented by \(A\) in the standard ordered basis, then \(U\) has two characteristic values, \(\iu\) and \(-\iu\). Here we see a subtle point. In discussing the characteristic values of a matrix \(A\), we must be careful to stipulate the field involved. The matrix \(A\) above has no characteristic values in \(R\), but has the two characteristic values \(\iu\) and \(-\iu\) in \(C\).
\end{example}

\begin{example}\label{example:6.2}
    Let \(A\) be the (real) \(3\times3\) matrix
    \begin{equation*}
        \begin{bmatrix}
            3 & 1 & -1 \\
            2 & 2 & -1 \\
            2 & 2 & 0
        \end{bmatrix}
        .
    \end{equation*}
    Then the characteristic polynomial for \(A\) is
    \begin{equation*}
        \begin{vmatrix}
            x-3 & -1 & 1 \\
            -2 & x-2 & 1 \\
            -2 & -2 & x
        \end{vmatrix}
        =x^3-5x^2+8x-4=\left(x-1\right)\left(x-2\right)^2.
    \end{equation*}
    Thus the characteristic values of \(A\) are \(1\) and \(2\).
    
    Suppose that \(T\) is the linear operator on \(R^3\) which is represented by \(A\) in the standard basis. Let us find the characteristic vectors of \(T\) associated with the characteristic values, \(1\) and \(2\). Now
    \begin{equation*}
        A-I=
        \begin{bmatrix}
            2 & 1 & -1 \\
            2 & 1 & -1 \\
            2 & 2 & -1
        \end{bmatrix}
        .
    \end{equation*}
    It is obvious at a glance that \(A-I\) has rank equal to \(2\) (and hence \(T-I\) has nullity equal to \(1\)). So the space of characteristic vectors associated with the characteristic value \(1\) is one-dimensional. The vector \(\alpha_1=\left(1,0,2\right)\) spans the null space of \(T-I\). Thus \(T\alpha=\alpha\) if and only if \(\alpha\) is a scalar multiple of \(\alpha_1\). Now consider
    \begin{equation*}
        A-2I=
        \begin{bmatrix}
            1 & 1 & -1 \\
            2 & 0 & -1 \\
            2 & 2 & -2
        \end{bmatrix}
        .
    \end{equation*}
    Evidently \(A-2I\) also has rank \(2\), so that the space of characteristic vectors associated with the characteristic value \(2\) has dimension \(1\). Evidently \(T\alpha=2\alpha\) if and only if \(\alpha\) is a scalar multiple of \(\alpha_2=\left(1,1,2\right)\).
\end{example}

\begin{definition}
    Let \(T\) be a linear operator on the finite-dimensional space \(V\). We say that \(T\) is \emph{diagonalizable} if there is a basis for \(V\) each vector of which is a characteristic vector of \(T\).
\end{definition}

The reason for the name should be apparent; for, if there is an ordered basis \(\symcal{B}=\set{\alpha_1,\ldots,\alpha_n}\) for \(V\) in which each \(\alpha_i\) is a characteristic vector of \(T\), then the matrix of \(T\) in the ordered basis \(\symcal{B}\) is diagonal. If \(T_{\alpha_i}=c_i\alpha_i\), then
\begin{equation*}
    \left[T\right]_{\symcal{B}}=
    \begin{bmatrix}
        c_1 & 0 & \cdots & 0 \\
        0 & c_2 & \cdots & 0 \\
        \vdots & \vdots && \vdots \\
        0 & 0 & \cdots & c_n
    \end{bmatrix}
    .
\end{equation*}
We certainly do not require that the scalars \(c_1,\ldots,c_n\) be distinct; indeed, they may all be the same scalar (when \(T\) is a scalar multiple of the identity operator).

One could also define \(T\) to be diagonalizable when the characteristic vectors of \(T\) span \(V\). This is only superficially different from our definition, since we can select a basis out of any spanning set of vectors.

For Examples~\ref{example:6.1} and \ref{example:6.2} we purposely chose linear operators \(T\) on \(R^n\) which are not diagonalizable. In Example~\ref{example:6.1}, we have a linear operator on \(R^2\) which is not diagonalizable, because it has no characteristic values. In Example~\ref{example:6.2}, the operator \(T\) has characteristic values; in fact, the characteristic polynomial for \(T\) factors completely over the real number field: \(f=\left(x-1\right)\left(x-2\right)^2\). Nevertheless \(T\) fails to be diagonalizable. There is only a one-dimensional space of characteristic vectors associated with each of the two characteristic values of \(T\). Hence, we cannot possibly form a basis for \(R^3\) which consists of characteristic vectors of \(T\).

Suppose that \(T\) is a diagonalizable linear operator. Let \(c_1\), \(\ldots\), \(c_k\) be the \emph{distinct} characteristic values of \(T\). Then there is an ordered basis \(\symcal{B}\) in which \(T\) is represented by a diagonal matrix which has for its diagonal entries the scalars \(c_i\), each repeated a certain number of times. If \(c_i\) is repeated \(d_i\) times, then (we may arrange that) the matrix has the block form
\begin{equation}
    \left[T\right]_{\symcal{B}}=
    \begin{bmatrix}
        c_1I_1 & 0 & \cdots & 0 \\
        0 & c_2I_2 & \cdots & 0 \\
        \vdots & \vdots && \vdots \\
        0 & 0 & \cdots & c_kI_k
    \end{bmatrix}
    \label{eq:6.3}
\end{equation}
where \(I_j\) is the \(d_j\times d_j\) identity matrix. From that matrix we see two things. First, the characteristic polynomial for \(T\) is the product of (possibly repeated) linear factors:
\begin{equation*}
    f=\left(x-c_1\right)^{d_1}\cdots\left(x-c_k\right)^{d_k}
\end{equation*}
If the scalar field \(F\) is algebraically closed, e.g., the field of complex numbers, every polynomial over \(F\) can be so factored (see Section~\ref{section:4.5}); however, if \(F\) is not algebraically closed, we are citing a special property of \(T\) when we say that its characteristic polynomial has such a factorization. The second thing we see from \eqref{eq:6.3} is that \(d_i\), the number of times which \(c_i\) is repeated as root of \(f\), is equal to the dimension of the space of characteristic vectors associated with the characteristic value \(c_i\). That is because the nullity of a diagonal matrix is equal to the number of zeros which it has on its main diagonal, and the matrix \(\left[T-c_iI\right]_{\symcal{B}}\) has \(d_i\) zeros on its main diagonal. This relation between the dimension of the characteristic space and the multiplicity of the characteristic value as a root of \(f\) does not seem exciting at first; however, it will provide us with a simpler way of determining whether a given operator is diagonalizable.

\begin{lemma}
    Suppose that \(T\alpha=c\alpha\). If \(f\) is any polynomial, then \(f\left(T\right)\alpha=f\left(c\right)\alpha\).
\end{lemma}

\begin{proof}
    Exercise.
\end{proof}

\begin{lemma}
    Let \(T\) be a linear operator on the finite-dimensional space \(V\). Let \(c_1\), \(\ldots\), \(c_k\) be the distinct characteristic values of \(T\) and let \(W_i\) be the space of characteristic vectors associated with the characteristic value \(c_i\). If \(W=W_1+\cdots+W_k\), then
    \begin{equation*}
        \dim W=\dim W_1+\cdots+\dim W_k.
    \end{equation*}
    In fact, if \(\symcal{B}_i\) is an ordered basis for \(W_i\), then \(\symcal{B}=\left(\symcal{B}_1,\ldots,\symcal{B}_k\right)\) is an ordered basis for \(W\).
\end{lemma}

\begin{proof}
    The space \(W=W_1+\cdots+W_k\) is the subspace spanned by all of the characteristic vectors of \(T\). Usually when one forms the sum \(W\) of subspaces \(W_i\), one expects that \(\dim W<\dim W_1+\cdots+\dim W_k\) because of linear relations which may exist between vectors in the various spaces. This lemma states that the characteristic spaces associated with different characteristic values are independent of one another.

    Suppose that (for each \(i\)) we have a vector \(\beta_i\) in \(W_i\), and assume that \(\beta_1+\cdots+\beta_k=0\). We shall show that \(\beta_i=0\) for each \(i\). Let \(f\) be any polynomial. Since \(T\beta_i=c_i\beta_i\), the preceding lemma tells us that
    \begin{align*}
        0=f\left(T\right)0&=f\left(T\right)\beta_1+\cdots+f\left(T\right)\beta_k\\
                          &=f\left(c_1\right)\beta_1+\cdots+f\left(c_k\right)\beta_k.
    \end{align*}
    Choose polynomials \(f_1\), \(\ldots\), \(f_k\) such that
    \begin{equation*}
        f_i\left(c_i\right)=\delta_{ij}=
        \begin{cases*}
            1, & \(i=j\) \\
            0, & \(i\ne j\).
        \end{cases*}
    \end{equation*}
    Then
    \begin{align*}
        0=f_i\left(T\right)0&=\sum_j\delta_{ij}\beta_j\\
                            &=\beta_i.
    \end{align*}
    Now, let \(\symcal{B}_i\) be an ordered basis for \(W_i\), and let \(\symcal{B}\) be the sequence \(\symcal{B}=\left(\symcal{B}_1,\ldots,\symcal{B}_k\right)\). Then \(\symcal{B}\) spans the subspace \(W=W_i+\cdots+W_k\). Also, \(\symcal{B}\) is a linearly independent sequence of vectors, for the following reason. Any linear relation between the vectors in \(\symcal{B}\) will have the form \(\beta_1+\cdots+\beta_k=0\), where \(\beta_i\) is some linear combination of the vectors in \(\symcal{B}_i\). From what we just did, we know that \(\beta_i=0\) for each \(i\). Since each \(\symcal{B}_i\) is linearly independent, we see that we have only the trivial linear relation between the vectors in \(\symcal{B}\).
\end{proof}

\begin{theorem}\label{theorem:6.2}
    Let \(T\) be a linear operator on a finite-dimensional space \(V\). Let \(c_1\), \(\ldots\), \(c_k\) be the distinct characteristic values of \(T\) and let \(W_i\) be the null space of \(\left(T-c_iI\right)\). The following are equivalent.
    \begin{enumerate}
        \item\label{itm:6.2.6} \(T\) is diagonalizable.
        \item\label{itm:6.2.7} The characteristic polynomial for \(T\) is
            \begin{equation*}
                f=\left(X-c_1\right)^{d_1}\cdots\left(X-c_k\right)^{d_k}
            \end{equation*}
            and \(\dim W_i=d_i\), \(i=1,\ldots,k\).
        \item\label{itm:6.2.8} \(\dim W_1+\cdots+\dim W_k=\dim V\).
    \end{enumerate}
\end{theorem}

\begin{proof}
    We have observed that \ref{itm:6.2.6} implies \ref{itm:6.2.7}. If the characteristic polynomial \(f\) is the product of linear factors, as in \ref{itm:6.2.7}, then \(d_1+\cdots+d_k=\dim V\). For, the sum of the \(d_i\)'s is the degree of the characteristic polynomial, and that degree is \(\dim V\). Therefore \ref{itm:6.2.7} implies \ref{itm:6.2.8}. Suppose \ref{itm:6.2.8} holds. By the lemma, we must have \(V=W_1+\cdots+W_k\), i.e., the characteristic vectors of \(T\) span \(V\).
\end{proof}

The matrix analogue of Theorem~\ref{theorem:6.2} may be formulated as follows. Let \(A\) be an \(n\times n\) matrix with entries in a field \(F\), and let \(c_1\), \(\ldots\), \(c_k\) be the distinct characteristic values of \(A\) in \(F\). For each \(i\), let \(W_i\) be the space of column matrices \(X\) (with entries in \(F\)) such that
\begin{equation*}
   \left(A-c_iI\right)X=0,
\end{equation*}
and let \(\symcal{B}_i\) be an ordered basis for \(W_i\). The bases \(\symcal{B}_1\), \(\ldots\), \(\symcal{B}_k\) collectively string together to form the sequence of columns of a matrix \(P\):
\begin{equation*}
    P=\left[P_1,P_2,\ldots\right]=\left(\symcal{B}_1,\ldots,\symcal{B}_k\right).
\end{equation*}
The matrix \(A\) is similar over \(F\) to a diagonal matrix if and only if \(P\) is a square matrix. When \(P\) is square, \(P\) is invertible and \(P^{-1}AP\) is diagonal.

\begin{example}
    Let \(T\) be the linear operator on \(R^3\) which is represented in the standard ordered basis by the matrix
    \begin{equation*}
        A=
        \begin{bmatrix}
            5 & -6 & -6 \\
            -1 & 4 & 2 \\
            3 & -6 & -4
        \end{bmatrix}
        .
    \end{equation*}
    Let us indicate how one might compute the characteristic polynomial, using various row and column operations:
    \begin{align*}
        \begin{vmatrix}
            x-5 & 6 & 6 \\
            1 & x-4 & -2 \\
            -3 & 6 & x+4
        \end{vmatrix}
        &=
        \begin{vmatrix}
            x-5 & 0 & 6 \\
            1 & x-2 & -2 \\
            -3 & 2-x & x+4
        \end{vmatrix}
        \\
        &=\left(x-2\right)
        \begin{vmatrix}
            x-5 & 0 & 6 \\
            1 & 1 & -2 \\
            -3 & -1 & x+4
        \end{vmatrix}
        \\
        &=\left(x-2\right)
        \begin{vmatrix}
            x-5 & 6 \\
            -2 & x+2
        \end{vmatrix}
        \\
        &=\left(x-2\right)\left(x^2-3x+2\right)\\
        &=\left(x-2\right)^2\left(x-1\right).
    \end{align*}
    What are the dimensions of the spaces of characteristic vectors associated with the two characteristic values? We have
    \begin{align*}
        A-I&=
        \begin{bmatrix}
            4 & -6 & -6 \\
            -1 & 3 & 2 \\
            3 & -6 & -5
        \end{bmatrix}
        \\
        A-2I&=
        \begin{bmatrix}
            3 & -6 & -6 \\
            -1 & 2 & 2 \\
            3 & -6 & -6
        \end{bmatrix}
        .
    \end{align*}
    We know that \(A-I\) is singular and obviously \(\rank\left(A-I\right)\geqslant2\). Therefore, \(\rank\left(A-I\right)=2\). It is evident that \(\rank\left(A-2I\right)=1\).

    Let \(W_1\), \(W_2\) be the spaces of characteristic vectors associated with the characteristic values \(1\), \(2\). We know that \(\dim W_1=1\) and \(\dim W_2=2\). By Theorem~\ref{theorem:6.2}, \(T\) is diagonalizable. It is easy to exhibit a basis for \(R^3\) in which \(T\) is represented by a diagonal matrix. The null space of \(\left(T-I\right)\) is spanned by the vector \(\alpha_1=\left(3,-1,3\right)\) and so \(\alpha_1\) is a basis for \(W_1\). The null space of \(T-2I\) (i.e., the space \(W_2\)) consists of the vectors \(\left(x_1,x_2,x_3\right)\) with \(x_1=2x_2+2x_3\). Thus, one example of a basis for \(W_2\) is
    \begin{align*}
        \alpha_2&=\left(2,1,0\right)\\
        \alpha_3&=\left(2,0,1\right).
    \end{align*}
    If \(\symcal{B}=\set{\alpha_1,\alpha_2,\alpha_3}\), then \(\left[T\right]_{\symcal{B}}\) is the diagonal matrix
    \begin{equation*}
        D=
        \begin{bmatrix}
            1 & 0 & 0 \\
            0 & 2 & 0 \\
            0 & 0 & 2
        \end{bmatrix}
        .
    \end{equation*}
    The fact that \(T\) is diagonalisable means that the original matrix \(A\) is similar (over \(R\)) to the diagonal matrix \(D\). The matrix \(P\) which enables us to change coordinates from the basis \(\symcal{B}\) to the standard basis is (of course) the matrix which has the transposes of \(\alpha_1\), \(\alpha_2\), \(\alpha_3\) as its column vectors:
    \begin{equation*}
        P=
        \begin{bmatrix}
            3 & 2 & 2 \\
            -1 & 1 & 0 \\
            3 & 0 & 1
        \end{bmatrix}
        .
    \end{equation*}
    Furthermore, \(AP=PD\), so that
    \begin{equation*}
        P^{-1}AP=D.
    \end{equation*}
\end{example}

\subsection*{Exercises}

\begin{enumerate}
    \item In each of the following cases, let \(T\) be the linear operator on \(R^2\) which is represented by the matrix \(A\) in the standard ordered basis for \(R^2\), and let \(U\) be the linear operator on \(C^2\) represented by \(A\) in the standard ordered basis. Find the characteristic polynomial for \(T\) and that for \(U\), find the characteristic values of each operator, and for each such characteristic value \(c\) find a basis for the corresponding space of characteristic vectors.
        \begin{equation*}
            A=
            \begin{bmatrix}
                1 & 0 \\
                0 & 0
            \end{bmatrix}
            ,\qquad A=
            \begin{bmatrix}
                2 & 3 \\
                -1 & 1
            \end{bmatrix}
            ,\qquad A=
            \begin{bmatrix}
                1 & 1 \\
                1 & 1
            \end{bmatrix}
            .
        \end{equation*}
    \item Let \(V\) be an \(n\)-dimensional vector space over \(F\). What is the characteristic polynomial of the identity operator on \(V\)? What is the characteristic polynomial for the zero operator?
    \item Let \(A\) be an \(n\times n\) triangular matrix over the field \(F\). Prove that the characteristic values of \(A\) are the diagonal entries of \(A\), i.e., the scalars \(A_{ii}\).
    \item Let \(T\) be the linear operator on \(R^3\) which is represented in the standard ordered basis by the matrix
        \begin{equation*}
            \begin{bmatrix}
                -9 & 4 & 4 \\
                -8 & 3 & 4 \\
                -16 & 8 & 7
            \end{bmatrix}
            .
        \end{equation*}
        Prove that \(T\) is diagonalizable by exhibiting a basis for \(R^3\), each vector of which is a characteristic vector of \(T\).
    \item Let
        \begin{equation*}
            A=
            \begin{bmatrix}
                6 & -3 & -2 \\
                4 & -1 & -2 \\
                10 & -5 & -3
            \end{bmatrix}
            .
        \end{equation*}
        Is \(A\) similar over the field \(R\) to a diagonal matrix? Is \(A\) similar over the field \(C\) to a diagonal matrix?
    \item Let \(T\) be the linear operator on \(R^4\) which is represented in the standard ordered basis by the matrix
        \begin{equation*}
            \begin{bmatrix}
                0 & 0 & 0 & 0 \\
                a & 0 & 0 & 0 \\
                0 & b & 0 & 0 \\
                0 & 0 & c & 0
            \end{bmatrix}
            .
        \end{equation*}
        Under what conditions on \(a\), \(b\), and \(c\) is \(T\) diagonalizable?
    \item Let \(T\) be a linear operator on the \(n\)-dimensional vector space \(V\), and suppose that \(T\) has \(n\) \emph{distinct} characteristic values. Prove that \(T\) is diagonalizable.
    \item\label{exercise:6.2.8} Let \(A\) and \(B\) be \(n\times n\) matrices over the field \(F\). Prove that if \(\left(I-AB\right)\) is invertible, then \(I-BA\) is invertible and
        \begin{equation*}
            \left(I-BA\right)^{-1}=I+B\left(I-AB\right)^{-1}A.
        \end{equation*}
    \item Use the result of Exercise~\ref{exercise:6.2.8} to prove that, if \(A\) and \(B\) are \(n\times n\) matrices over the field \(F\), then \(AB\) and \(BA\) have precisely the same characteristic values in \(F\).
    \item Suppose that \(A\) is a \(2\times2\) matrix with real entries which is symmetric (\(A^\transpose=A\)). Prove that \(A\) is similar over \(R\) to a diagonal matrix.
    \item\label{exercise:6.2.11} Let \(N\) be a \(2\times2\) complex matrix such that \(N^2=0\). Prove that either \(N=0\) or \(N\) is similar over \(C\) to
        \begin{equation*}
            \begin{bmatrix}
                0 & 0 \\
                1 & 0
            \end{bmatrix}
            .
        \end{equation*}
    \item Use the result of Exercise~\ref{exercise:6.2.11} to prove the following: If \(A\) is a \(2\times2\) matrix with complex entries, then \(A\) is similar over \(C\) to a matrix of one of the two types
        \begin{equation*}
            \begin{bmatrix}
                a & 0 \\
                0 & b
            \end{bmatrix}
            \qquad
            \begin{bmatrix}
                a & 0 \\
                1 & a
            \end{bmatrix}
            .
        \end{equation*}
    \item Let \(V\) be the vector space of all functions from \(R\) into \(R\) which are continuous, i.e., the space of continuous real-valued functions on the real line. Let \(T\) be the linear operator on \(V\) defined by
        \begin{equation*}
            \left(Tf\right)\left(x\right)=\int_0^xf\left(t\right)\odif{t}.
        \end{equation*}
        Prove that \(T\) has no characteristic values.
    \item Let \(A\) be an \(n\times n\) diagonal matrix with characteristic polynomial
        \begin{equation*}
            \left(x-c_1\right)^{d_1}\cdots\left(x-c_k\right)^{d_k},
        \end{equation*}
        where \(c_1\), \(\ldots\), \(c_k\) are distinct. Let \(V\) be the space of \(n\times n\) matrices \(B\) such that \(AB=BA\). Prove that the dimension of \(V\) is \(d_1^2+\cdots+d_k^2\).
    \item Let \(V\) be the space of \(n\times n\) matrices over \(F\). Let \(A\) be a fixed \(n\times n\) matrix over \(F\). Let \(T\) be the linear operator `left multiplication by \(A\)' on \(V\). Is it true that \(A\) and \(T\) have the same characteristic values?
\end{enumerate}



\section{Constraints}

From the previous sections one might obtain the impression that all problems in mechanics have been reduced to solving the set of differential equations~\eqref{eq:1.19}:
\begin{equation*}
    m_i\ddot{\symbf{r}}_i=\symbf{F}_i^{\left(e\right)}+\sum_j\symbf{F}_{ji}.
\end{equation*}
One merely substitutes the various forces acting upon the particles of the system, turns the mathematical crank, and grinds out the answers! Even from a purely physical standpoint, however, this view is oversimplified. For example, it may be necessary to take into account the \emph{constraints} that limit the motion of the system. We have already met one type of system involving constraints, namely rigid bodies, where the constraints on the motions of the particles keep the distances \(r_{ij}\) unchanged. Other examples of constrained systems can easily be furnished. The beads of an abacus are constrained to one-dimensional motion by the supporting wires. Gas molecules within a container are constrained by the walls of the vessel to move only \emph{inside} the container. A particle placed on the surface of a solid sphere is subject to the constraint that it can move only on the surface or in the region exterior to the sphere.

Constraints may be classified in various ways, and we shall use the following system. If the conditions of constraint can be expressed as equations connecting the coordinates of the particles (and possibly the time) having the form
\begin{equation}
    f\left(\symbf{r}_1,\symbf{r}_2,\symbf{r}_3,\ldots,t\right)=0,\label{eq:1.37}
\end{equation}
then the constraints are said to be \emph{holonomic}. Perhaps the simplest example of holonomic constraints is the rigid body, where the constraints are expressed by equations of the form
\begin{equation*}
    \left(\symbf{r}_i-\symbf{r}_j\right)^2-c_{ij}^2=0.
\end{equation*}
A particle constrained to move along any curve or on a given surface is another obvious example of a holonomic constraint, with the equations defining the curve or surface acting as the equations of a constraint.

Constraints not expressible in this fashion are called nonholonomic. The walls of a gas container constitute a nonholonomic constraint. The constraint involved in the example of a particle placed on the surface of a sphere is also nonholonomic, for it can be expressed as an inequality
\begin{equation*}
    r^2-a^2\geqslant0
\end{equation*}
(where \(a\) is the radius of the sphere), which is not in the form of \eqref{eq:1.37}. Thus, in a gravitational field a particle placed on the top of the sphere will slide down the surface part of the way but will eventually fall off.

Constraints are further classified according to whether the equations of constraint contain the time as an explicit variable (rheonomous) or are not explicitly dependent on time (scleronomous). A bead sliding on a rigid curved wire fixed in space is obviously subject to a scleronomous constraint; if the wire is moving in some prescribed fashion, the constraint is rheonomous. Note that if the wire moves, say, as a reaction to the bead's motion, then the time dependence of the constraint enters in the equation of the constraint only through the coordinates of the curved wire (which are now part of the system coordinates). The overall constraint is then scleronomous.

Constraints introduce two types of difficulties in the solution of mechanical problems. First, the coordinates \(r_i\) are no longer all independent, since they are connected by the equations of constraint; hence the equations of motion \eqref{eq:1.19} are not all independent. Second, the forces of constraint, e.g., the force that the wire exerts on the bead (or the wall on the gas particle), is not furnished a priori. They are among the unknowns of the problem and must be obtained from the solution we seek. Indeed, imposing constraints on the system is simply another method of stating that there are forces present in the problem that cannot be specified directly but are known rather in terms of their effect on the motion of the system.

In the case of holonomic constraints, the first difficulty is solved by the introduction of \emph{generalized coordinates}. So far we have been thinking implicitly in terms of Cartesian coordinates. A system of \(N\) particles, free from constraints, has \(3N\) independent coordinates or \emph{degrees of freedom}. If there exist holonomic constraints, expressed in \(k\) equations in the form \eqref{eq:1.37}, then we may use these equations to eliminate \(k\) of the \(3N\) coordinates, and we are left with \(3N-k\) independent coordinates, and the system is said to have \(3N-k\) degrees of freedom. This elimination of the dependent coordinates can be expressed in another way, by the introduction of new, \(3N-k\), independent variables \(q_1\), \(q_2\), \ldots, \(q_{3N-k}\) in terms of which the old coordinates \(\symbf{r}_1\), \(\symbf{r}_2\), \ldots, \(\symbf{r}_N\) are expressed by equations of the form
\begin{equation}
    \begin{aligned}
        \symbf{r}_1&=\symbf{r}_1\left(q_1,q_2,\ldots,q_{3N-k},t\right),\\
        &\vdotswithin{=}\\
        \symbf{r}_N&=\symbf{r}_N\left(q_1,q_2,\ldots,q_{3N-k},t\right)
    \end{aligned}
    \label{eq:1.38}
\end{equation}
containing the constraints in them implicitly. These are \emph{transformation} equations from the set of \(\left(\symbf{r}_l\right)\) variables to the \(\left(q_l\right)\) set, or alternatively Eqs.~\eqref{eq:1.38} can be considered as parametric representations of the \(\left(\symbf{r}_l\right)\) variables. It is always assumed that we can also transform back from the \(\left(q_l\right)\) to the \(\left(\symbf{r}_l\right)\) set, i.e., that Eqs.~\eqref{eq:1.38} combined with the \(k\) equations of constraint can be inverted to obtain any \(q_i\) as a function of the \(\left(\symbf{r}_l\right)\) variable and time.

Usually the generalized coordinates, \(q_l\), unlike the Cartesian coordinates, will not divide into convenient groups of three that can be associated together to form vectors. Thus, in the case of a particle constrained to move \emph{on} the surface of a sphere, the two angles expressing position on the sphere, say latitude and longtude, are obvious possible generalized coordinates. Or, in the example of a double pendulum moving in a plane (two particles connected by an inextensible light rod and suspended by a similar rod fastened to one of the particles), satisfactory generalized coordinates are the two angles \(\theta_1\), \(\theta_2\). (Cf. Fig.~\ref{fig:1.4}.) Generalized coordinates, in the sense of coordinates other than Cartesian, are often useful in systems without constraints. Thus, in the problem of a particle moving in an external central force field (\(V=V\left(r\right)\)), there is no constraint involved, but it is clearly more convenient to use spherical polar coordinates than Cartesian coordinates. Do not, however, think of generalized coordinates in terms of conventional orthogonal position coordinates. All sorts of quantities may be invoked to serve as generalized coordinates. Thus, the amplitudes in a Fourier expansion of \(\symbf{r}_j\) may be used as generalized coordinates, or we may find it convenient to employ quantities with the dimensions of energy or angular momentum.

\begin{figure}[htbp]
    \centering
    \includegraphics[scale = 0.225]{01/figures/1.4}
    \caption{Double pendulum.}
    \label{fig:1.4}
\end{figure}

If the constraint is nonholonomic, the equations expressing the constraint cannot be used to eliminate the dependent coordinates. An oft-quoted example of a nonholonomic constraint is that of an object rolling on a rough surface without slipping. The coordinates used to describe the system will generally involve angular coordinates to specify the orientation of the body, plus a set of coordinates describing the location of the point of contact on the surface. The constraint of ``rolling" connects these two sets of coordinates; they are not independent. A change in the position of the point of contact inevitably means a change in its orientation. Yet we cannot reduce the number of coordinates, for the ``rolling" condition is not expressible as a equation between the coordinates, in the manner of \eqref{eq:1.37}. Rather, it is a condition on the \emph{velocities} (i.e., the point of contact is stationary), a differential condition that can be given in an integrated form only \emph{after} the problem is solved.

A simple case will illustrate the point. Consider a disk rolling on the horizontal \(xy\) plane constrained to move so that the plane of the disk is always vertical. The coordinates used to describe the motion might be the \(x\), \(y\) coordinates of the center of the disk, an angle of rotation \(\phi\) about the axis of the disk, and an angle \(\theta\) between the axis of the disk and say, the \(x\) axis (cf. Fig~\ref{fig:1.5}). As a result of the constraint the velocity of the center of the disk, \(\symbf{v}\), has a magnitude proportional to \(\dot{\phi}\),
\begin{equation*}
    v=a\dot{\phi},
\end{equation*}
where \(a\) is the radius of the disk, and its direction is perpendicular to the axis of the disk:
\begin{equation*}
    \begin{aligned}
        \dot{x}&=v\sin\theta,\\
        \dot{y}&=-v\cos\theta.
    \end{aligned}
\end{equation*}
Combining these conditions, we have two \emph{differential} equations of constraint:
\begin{equation}
    \begin{aligned}
        \odif{x}-a\sin\theta\odif{\phi}&=0,\\
        \odif{y}+a\cos\theta\odif{\phi}&=0.
    \end{aligned}
    \label{eq:1.39}
\end{equation}
Neither of Eqs.~\eqref{eq:1.39} can be integrated without in fact solving the problem; i.e., we cannot find an integrating factor \(f\left(x,y,\theta,\phi\right)\) that will turn either of the equations into perfect differentials (cf. Derivation~\ref{derivation:1.4}).\footnote{In principle, an integrating factor can always be found for a first-order differential equation of constraint in systems involving only two coordinates and such constraints are therefore holonomic. A familiar example is the two-dimensional motion of a circle rolling on an inclined plane.} Hence, the constraints cannot be reduced to the form of Eq.~\eqref{eq:1.37} and are therefore nonholonomic. Physically we can see that there can be no direct functional relation between \(\phi\) and the other coordinates \(x\), \(y\), and \(\theta\) by noting that at any point on its path the disk can be made to roll around in a circle tangent to the path and of arbitrary radius. At the end of the process, \(x\), \(y\), and \(\theta\) have been returned to their original values, but \(\phi\) has changed by an amount depending on the radius of the circle.

\begin{figure}[htbp]
    \centering
    \includegraphics[scale = 0.225]{01/figures/1.5}
    \caption{Vertical disk rolling on a horizontal plane.}
    \label{fig:1.5}
\end{figure}

Nonintegrable \emph{differential} constraints of the form of Eqs.~\eqref{eq:1.39} are of course not the only type of nonholonomic constraints. The constraint conditions may involve higher-order derivatives, or may appear in the form of inequalities, as we have seen.

Partly because the dependent coordinates can be eliminated, problems involying holonomic constraints are always amenable to a formal solution. But there is no general way to attack nonholonomic examples. True, if the constraint is nonintegrable, the differential equations of constraint can be introduced into the problem along with the differential equations of motion, and the dependent equations eliminated, in effect, by the method of Lagrange multipliers.

We shall return to this method at a later point. However, the more vicious cases of nonholonomic constraint must be tackled individually, and consequently in the development of the more formal aspects of classical mechanics, it is almost invariably assumed that any constraint, if present, is holonomic. This restriction does not greatly limit the applicability of the theory, despite the fact that many of the constraints encountered in everyday life are nonholonomic. The reason is that the entire concept of constraints imposed in the system through the medium of wires or surfaces or walls is particularly appropriate only in macroscopic or large-scale problems. But today physicists are more interested in atomic and nuclear problems. On this scale all objects, both in and out of the system, consist alike of molecules, atoms, or smaller particles, exerting definite forces, and the notion of constraint becomes artificial and rarely appears. Constraints are then used only as mathematical idealizations to the actual physical case or as classical approximations to a quantum-mechanical property, e.g., rigid body rotations for ``spin." Such constraints are always holonomic and fit smoothly into the framework of the theory.

To surmount the second difficulty, namely, that the forces of constraint are unknown a priori, we should like to so formulate the mechanics that the forces of constraint disappear. We need then deal only with the known applied forces. A hint as to the procedure to be followed is provided by the fact that in a particular system with constraints, i.e., a rigid body, the work done by internal forces (which are here the forces of constraint) vanishes. We shall follow up this clue in the ensuing sections and generalize the ideas contained in it.


\section{D'Alembert's Principle and Lagrange's Equations}

A virtual (infinitesimal) displacement of a system refers to a change in the configuration of the system as the result of any arbitrary infinitesimal change of the coordinates \(\fdif{\symbf{r}_i}\), \emph{consistent with the forces and constraints imposed on the system at the given instant \(t\)}. The displacement is called virtual to distinguish it from an actual displacement of the system occurring in a time interval \(\odif{t}\), during which the forces and constraints may be changing. Suppose the system is in equilibrium; i.e., the total force on each particle vanishes, \(\symbf{F}_i=0\). Then clearly the dot product \(\symbf{F}_i\cdot\fdif{\symbf{r}_i}\), which is the virtual work of the force \(\symbf{F}_i\) in the displacement \(\fdif{\symbf{r}_i}\), also vanishes. The sum of these vanishing products over all particles must likewise be zero:
\begin{equation}
    \sum_i\symbf{F}_i\cdot\fdif{\symbf{r}_i}=0.\label{eq:1.40}
\end{equation}
As yet nothing has been said that has any new physical content. Decompose \(\symbf{F}_i\) into the applied force, \(\symbf{F}_i^{\left(a\right)}\), and the force of constraint, \(\symbf{f}_i\),
\begin{equation}
    \symbf{F}_i=\symbf{F}_i^{\left(a\right)}+\symbf{f}_i,
\end{equation}
so that Eq.~\eqref{eq:1.40} becomes
\begin{equation}
    \sum_i\symbf{F}_i^{\left(a\right)}\cdot\fdif{\symbf{r}_i}+\sum_i\symbf{f}_i\cdot\fdif{\symbf{r}_i}=0.
\end{equation}
We now restrict ourselves to systems for which \emph{the net virtual work of the forces of constraint is zero}. We have seen that this condition holds true for rigid bodies and it is valid for a large number of other constraints. Thus, if a particle is constrained to move on a surface, the force of constraint is perpendicular to the surface, while the virtual displacement must be tangent to it, and hence the virtual work vanishes. This is no longer true if sliding friction forces are present, and we must exclude such systems from our formulation. The restriction is not unduly hampering, since the friction is essentially a macroscopic phenomenon. On the other hand, the forces of rolling friction do not violate this condition, since the forces act on a point that is momentarily at rest and can do no work in an infinitesimal displacement consistent with the rolling constraint. Note that if a particle is constrained to a surface that is itself moving in time, the force of constraint is instantaneously perpendicular to the surface and the work during a virtual displacement is still zero even though the work during an actual displacement in the time \(\odif{t}\) does not necessarily vanish.

We therefore have as the condition for equilibrium of a system that the virtual work of the \emph{applied forces} vanishes:
\begin{equation}
    \sum_i\symbf{F}_i^{\left(a\right)}\cdot\fdif{\symbf{r}_i}=0.\label{eq:1.43}
\end{equation}
Equation~\eqref{eq:1.43} is often called the \emph{principle of virtual work}. Note that the coefficients of \(\fdif{\symbf{r}_i}\) can no longer be set equal to zero; 1.e., in general \(\symbf{F}_i^{\left(a\right)}\ne0\), since the \(\fdif{\symbf{r}_i}\) are not completely independent but are connected by the constraints. In order to equate the coefficients to zero, we must transform the principle into a form involving the virtual displacements of the \(q_i\), which are independent. Equation \eqref{eq:1.43} satisfies our needs in that it does not contain the \(\symbf{f}_i\), but it deals only with statics; we want a condition involving the general motion of the system.

To obtain such a principle, we use a device first thought of by James Bernoulli and developed by D'Alembert. The equation of motion,
\begin{equation*}
    \symbf{F}_i=\dot{\symbf{p}}_i,
\end{equation*}
can be written as
\begin{equation*}
    \symbf{F}_i-\dot{\symbf{p}}_i=0,
\end{equation*}
which states that the particles in the system will be in equilibrium under a force equal to the actual force plus a ``reversed effective force" \(-\dot{\symbf{p}}_i\). Instead of \eqref{eq:1.40}, we can immediately write
\begin{equation}
    \sum_i\left(\symbf{F}-\dot{\symbf{p}}_i\right)\cdot\fdif{\symbf{r}_i}=0,
\end{equation}
and, making the same resolution into applied forces and forces of constraint, there results
\begin{equation*}
    \sum_i\left(\symbf{F}_i^{\left(a\right)}-\dot{\symbf{p}}_i\right)\cdot\fdif{\symbf{r}_i}+\sum_i\symbf{f}_i\cdot\fdif{\symbf{r}_i}=0.
\end{equation*}
We again restrict ourselves to systems for which the virtual work of the forces of constraint vanishes and therefore obtain
\begin{equation}
    \sum_i\left(\symbf{F}_i^{\left(a\right)}-\dot{\symbf{p}}_i\right)\cdot\fdif{\symbf{r}_i}=0,\label{eq:1.45}
\end{equation}
which is often called \emph{D'Alembert's principle}. We have achieved our aim, in that the forces of constraint no longer appear, and the superscript \(^{\left(a\right)}\) can now be dropped without ambiguity. It is still not in a useful form to furnish equations of motion for the system. We must now transform the principle into an expression involving virtual displacements of the generalized coordinates, which are then independent of each other (for holonomic constraints), so that the coefficients of the \(\fdif{q_i}\) can be set separately equal to zero.

The translation from \(\symbf{r}_i\) to \(q_j\) language starts from the transformation equations~\eqref{eq:1.38},
\begin{equation}
    \symbf{r}_i=\symbf{r}_i\left(q_1,q_2,\ldots,q_n,t\right)\tag{\ref{eq:1.45}\prime}
\end{equation}
(assuming \(n\) independent coordinates), and is carried out by means of the usual ``chain rules" of the calculus of partial differentiation. Thus, \(\symbf{v}_i\) is expressed in terms of the \(\dot{q}_k\) by the formula
\begin{equation}
    \symbf{v}_i\equiv\odv{\symbf{r}_i}{t}=\sum_k\pdv{\symbf{r}_i}{q_k}\dot{q}_k+\pdv{\symbf{r}_i}{t}.\label{eq:1.46}
\end{equation}
Similarly, the arbitrary virtual displacement \(\fdif{\symbf{r}_i}\) can be connected with the virtual displacements \(\fdif{q_i}\) by
\begin{equation}
    \fdif{\symbf{r}_i}=\sum_j\pdv{\symbf{r}_i}{q_j}\fdif{q_j}.\label{eq:1.47}
\end{equation}
Note that no variation of time, \(\fdif{t}\), is involved here, since a virtual displacement by definition considers only displacements of the coordinates. (Only then is the virtual displacement perpendicular to the force of constraint if the constraint itself is changing in time.)

In terms of the generalized coordinates, the virtual work of the \(\symbf{F}_i\) becomes
\begin{equation}
    \begin{aligned}[t]
        \sum_i\symbf{F}_i\cdot\fdif{\symbf{r}_i}&=\sum_{i,j}\symbf{F}_i\cdot\pdv{\symbf{r}_i}{q_j}\fdif{q_j}\\
        &=\sum_jQ_j\fdif{q_j},
    \end{aligned}
\end{equation}
where the \(Q_j\) are called the components of the \emph{generalized force}, defined as
\begin{equation}
    Q_j=\sum_i\symbf{F}_i\cdot\pdv{\symbf{r}_i}{q_j}.\label{eq:1.49}
\end{equation}
Note that just as the \(q\)'s need not have the dimensions of length, so the \(Q\)'s do not necessarily have the dimensions of force, but \(Q_j\fdif{q_j}\) must always have the dimensions of work. For example, \(Q_j\) might be a torque \(N_j\) and \(\odif{q_j}\) a differential angle \(\odif{\theta_j}\), which makes \(N_j\odif{\theta_j}\) a differential of work.

We turn now to the other other term involved in Eq.~\eqref{eq:1.45}, which may be written as
\begin{equation*}
    \sum_i\dot{\symbf{p}}_i\cdot\fdif{\symbf{r}_i}=\sum_im_i\ddot{\symbf{r}}_i\cdot\fdif{\symbf{r}_i}.
\end{equation*}
Expressing \(\fdif{\symbf{r}_i}\) by \eqref{eq:1.47}, this becomes
\begin{equation*}
    \sum_{i,j}m_i\ddot{\symbf{r}}_i\cdot\pdv{\symbf{r}_i}{q_j}\fdif{q_j}.
\end{equation*}
Consider now the relation
\begin{equation}
    \sum_im_i\ddot{\symbf{r}}_i\cdot\pdv{\symbf{r}_i}{q_j}=\sum_i\left[\odv*{\left(m_i\dot{\symbf{r}}_i\cdot\pdv{\symbf{r}_i}{q_j}\right)}{t}-m_i\dot{\symbf{r}}_i\cdot\odv*{\left(\pdv{\symbf{r}_i}{q_j}\right)}{t}\right].\label{eq:1.50}
\end{equation}
In the last term of Eq.~\eqref{eq:1.50} we can interchange the differentiation with respect to \(t\) and \(q_j\), for, in analogy to \eqref{eq:1.46},
\begin{equation*}
    \begin{aligned}
        \odv*{\left(\pdv{\symbf{r}_i}{q_j}\right)}{t}&=\pdv{\dot{\symbf{r}}_i}{q_j}=\sum_k\pdv{\symbf{r}_i}{q_j,q_k}\dot{q}_k+\pdv{\symbf{r}_i}{q_j,t},\\
        &=\pdv{\symbf{v}_i}{q_j},
    \end{aligned}
\end{equation*}
by Eq.~\eqref{eq:1.46}. Further, we also see from Eq.~\eqref{eq:1.46} that
\begin{equation}
    \pdv{\symbf{v}_i}{\dot{q}_j}=\pdv{\symbf{r}_i}{q_j}.
\end{equation}
Substitution of these changes in \eqref{eq:1.50} leads to the result that
\begin{equation*}
    \sum_im_i\ddot{\symbf{r}}_i\cdot\pdv{\symbf{r}_i}{q_j}=\sum_i\left[\odv*{\left(m_i\symbf{v}_i\cdot\pdv{\symbf{v}_i}{\dot{q}_j}\right)}{t}-m_i\symbf{v}_i\cdot\pdv{\symbf{v}_i}{q_j}\right],
\end{equation*}
and the second term on the left-hand side of Eq.~\eqref{eq:1.45} can be expanded into
\begin{equation*}
    \sum_j\left\{\odv*{\left[\pdv*{\left(\sum_i\frac{1}{2}m_iv_i^2\right)}{\dot{q}_j}\right]}{t}-\pdv*{\left(\sum_i\frac{1}{2}m_iv_i^2\right)}{q_j}\right\}\fdif{q_j}.
\end{equation*}
Identifying \(\sum_i\frac{1}{2}m_iv_i^2\) with the system kinetic energy \(T\), D'Alembert's principle (cf. Eq.~\eqref{eq:1.45}) becomes
\begin{equation}
    \sum\left\{\left[\odv*{\left(\pdv{T}{\dot{q}_j}\right)}{t}-\pdv{T}{q_j}\right]-Q_j\right\}\fdif{q_j}=0.\label{eq:1.52}
\end{equation}
Note that in a system of Cartesian coordinates the partial derivative of \(T\) with respect to \(q_j\) vanishes. Thus, speaking in the language of differential geometry, this term arises from the curvature of the coordinates \(q_j\). In polar coordinates, e.g., it is in the partial derivative of \(T\) with respect to an angle coordinate that the centripetal acceleration term appears.

Thus far, no restriction has been made on the nature of the constraints other than that they be workless in a virtual displacement. The variables \(q_j\) can be any set of coordinates used to describe the motion of the system. If, however, the constraints are holonomic, then it is possible to find sets of independent coordinates \(q_j\) that contain the constraint conditions implicitly in the transformation equations \eqref{eq:1.38}. Any virtual displacement \(\fdif{q_j}\) is then independent of \(\fdif{q_k}\), and therefore the only way for \eqref{eq:1.52} to hold is for the individual coefficients to vanish:
\begin{equation}
    \odv*{\left(\pdv{T}{\dot{q}_j}\right)}{t}-\pdv{T}{q_j}=Q_j.\label{eq:1.53}
\end{equation}
There are \(n\) such equations in all.

When the forces are derivable from a scalar potential function \(V\),
\begin{equation*}
    \symbf{F}_i=-\symbf{\nabla}_iV.
\end{equation*}
Then the generalized forces can be written as
\begin{equation*}
    Q_j=\sum_i\symbf{F}_i\cdot\pdv{\symbf{r}_i}{q_i}=-\sum_i\symbf{\nabla}_iV\cdot\pdv{\symbf{r}_i}{q_j},
\end{equation*}
which is exactly the same expression for the partial derivative of a function \(-V\left(\symbf{r}_1,\symbf{r}_2,\ldots,\symbf{r}_N,t\right)\) with respect to \(q_j\):
\begin{equation}
    Q_j=-\pdv{V}{q_j}.
\end{equation}
Equations~\eqref{eq:1.53} can then be rewritten as
\begin{equation}
    \odv*{\left(\pdv{T}{\dot{q}_j}\right)}{t}-\pdv{\left(T-V\right)}{q_j}=0.\label{eq:1.55}
\end{equation}
The equations of motion in the form \eqref{eq:1.55} are not necessarily restricted to conservative systems; only if \(V\) is not an explicit function of time is the system conservative (cf. p. \pageref{anchor:1.1}). As here defined, the potential \(V\) does not depend on the generalized velocities. Hence, we can include a term in \(V\) in the partial derivative with respect to \(\dot{q}_j\):
\begin{equation*}
    \odv*{\left(\pdv{\left(T-V\right)}{\dot{q}_j}\right)}{t}-\pdv{\left(T-V\right)}{q_j}=0.
\end{equation*}
Or, defining a new function, the \emph{Lagrangian} \(L\), as
\begin{equation}
    L=T-V,\label{eq:1.56}
\end{equation}
the Eqs.~\eqref{eq:1.53} become
\begin{equation}
    \odv*{\left(\pdv{L}{\dot{q}_j}\right)}{t}-\pdv{L}{q_j}=0.\label{eq:1.57}
\end{equation}
expressions referred to as ``Lagrange's equations."

Note that for a particular set of equations of motion there is no unique choice of Lagrangian such that Eqs.~\eqref{eq:1.57} lead to the equations of motion in the given generalized coordinates. Thus, in Derivations~\ref{derivation:1.8} and \ref{derivation:1.10} it is shown that if \(L\left(q,\dot{q},t\right)\) is an approximate Lagrangian and \(F\left(q,t\right)\) is \emph{any} differentiable function of the generalized coordinates and time, then
\begin{equation}
    L'\left(q,\dot{q},t\right)=L\left(q,\dot{q},t\right)+\odv{F}{t}\tag{\ref{eq:1.57}\prime}\label{eq:1.57'}
\end{equation}
is a Lagrangian also resulting in the same equations of motion. It is also often possible to find alternative Lagrangians beside those constructed by this prescription (see Exercise~\ref{exercise:1.20}). While Eq.~\eqref{eq:1.56} is always a suitable way to construct a Lagrangian for a conservative system, it does not provide the only Lagrangian suitable for the given system.


\section*{Derivations}

\begin{enumerate}
    \item Show that for a single particle with constant mass the equation of motion implies the following differential equation for the kinetic energy:
    \begin{equation*}
        \odv{T}{t}=\symbf{F}\cdot\symbf{v},
    \end{equation*}
    while if the mass varies with time the corresponding equation is
    \begin{equation*}
        \odv{\left(mT\right)}{t}=\symbf{F}\cdot\symbf{p}.
    \end{equation*}
    \item Prove that the magnitude \(R\) of the position vector for the center of mass from an arbitrary origin is given by the equation
    \begin{equation*}
        M^2R^2=M\sum_im_ir_i^2-\frac{1}{2}\sum_{i,j}m_im_jr_{ij}^2.
    \end{equation*}
    \item Suppose a system of two particles is known to obey the equations of motion, Eqs.~\eqref{eq:1.22} and \eqref{eq:1.26}. From the equations of the motion of the individual particles show that the internal forces between particles satisfy both the weak and the strong laws of action and reaction. The argument may be generalized to a system with arbitrary number of particles, thus proving the converse of the arguments leading to Eqs.~\eqref{eq:1.22} and \eqref{eq:1.26}.
    \item\label{derivation:1.4} The equations of constraint for the rolling disk, Eqs.~\eqref{eq:1.39}, are special cases of gen- eral linear differential equations of constraint of the form
    \begin{equation*}
        \sum_{i=1}^ng_i\left(x_1,\ldots,x_n\right)\odif{x_i}=0.
    \end{equation*}
    A constraint condition of this type is holonomic only if an integrating function \(f\left(x_1,\ldots,x_n\right)\) can be found that turns it into an exact differential. Clearly the function must be such that
    \begin{equation*}
        \pdv{\left(fg_i\right)}{x_j}=\pdv{\left(fg_j\right)}{x_i}
    \end{equation*}
    for all \(i\ne j\). Show that no such integrating factor can be found for either of Eqs.~\eqref{eq:1.39}.
    \item Two wheels of radius \(a\) are mounted on the ends of a common axle of length \(b\) such that the wheels rotate independently. The whole combination rolls without slipping on a plane. Show that there are two nonholonomic equations of constraint,
    \begin{equation*}
        \begin{aligned}
            \cos\theta\odif{x}+\sin\theta\odif{y}&=0,\\
            \sin\theta\odif{x}-\cos\theta\odif{y}&=\frac{1}{2}a\left(\odif{\phi}+\odif{\phi'}\right),
        \end{aligned}
    \end{equation*}
    (where \(\theta\), \(\phi\), and \(\phi'\) have meanings similar to those in the problem of a single vertical disk, and \(\left(x,y\right)\) are the coordinates of a point on the axle midway between the two wheels) and one holonomic equation of constraint,
    \begin{equation*}
        \theta=C-\frac{a}{b}\left(\phi-\phi'\right).
    \end{equation*}
    where \(C\) is a constant.
    \item A particle moves in the \(xy\) plane under the constraint that its velocity vector is always directed towards a point on the \(x\) axis whose abscissa is some given function of time \(f\left(t\right)\). Show that for \(f\left(t\right)\) differentiable, but otherwise arbitrary, the constraint is nonholonomic.
    \item Show that Lagrange's equations in the form of Eqs.~\eqref{eq:1.53} can also be written as
    \begin{equation*}
        \pdv{\dot{T}}{\dot{q}_j}-2\pdv{T}{q_j}=Q_j.
    \end{equation*}
    These are sometimes known as the \emph{Nielsen} form of the Lagrange equations.
    \item\label{derivation:1.8} If \(L\) is a Lagrangian for a system of \(n\) degrees of freedom satisfying Lagrange's equations, show by direct substitution that
    \begin{equation*}
        L'=L+\odv{F\left(q_1,\ldots,q_n,t\right)}{t}
    \end{equation*}
    also satisfies Lagrange's equations where \(F\) is any arbitrary, but differentiable, function of its arguments.
    \item The electromagnetic field is invariant under a gauge transformation of the scalar and vector potential given by
    \begin{equation*}
        \begin{aligned}
            \symbf{A}&\to\symbf{A}+\symbf{\nabla}\psi\left(\symbf{r},t\right),\\
            \phi&\to\phi-\frac{1}{c}\pdv{\psi}{t},
        \end{aligned}
    \end{equation*}
    where \(\psi\) is arbitrary (but differentiable). What effect does this gauge transformation have on the Lagrangian of a particle moving in the electromagnetic field? Is the motion affected?
    \item\label{derivation:1.10} Let \(q_1\), \(\ldots\), \(q_n\) be a set of independent generalized coordinates for a system of \(n\) degrees of freedom, with a Lagrangian \(L\left(q,\dot{q},t\right)\). Suppose we transform to another set of independent coordinates \(s_1\), \(\ldots\), \(s_n\) by means of transformation equations
    \begin{equation*}
        q_i=q_i\left(s_1,\ldots,s_n,t\right),\qquad i=1,\ldots,n.
    \end{equation*}
    (Such a transformation is called a \emph{point transformation}.) Show that if the Lagrangian function is expressed as a function of \(s_j\), \(\dot{s}_j\), and \(t\) through the equations of transformation, then \(L\) satisfies Lagrange's equations with respect to the \(s\) coordinates:
    \begin{equation*}
        \odv*{\left(\pdv{L}{\dot{s}_j}\right)}{t}-\pdv{L}{s_j}=0.
    \end{equation*}
    In other words, the form of Lagrange's equations is invariant under a point transformation. 
\end{enumerate}


\section{Exercises}

\begin{enumerate}
    \item Prove that the empty set is a subset of every set.
    \item\label{exercise:2.2} A complex number \(z\) is said to be \emph{algebraic} if there are integers \(a_0\), \(\ldots\), \(a_n\), not all zero, such that
    \begin{equation*}
        a_0z^n+a_1z^{n-1}+\cdots+a_{n-1}z+a_n=0.
    \end{equation*}
    Prove that the set of all algebraic numbers is countable. \hint{For every positive integer \(N\) there are only finitely many equations with
    \begin{equation*}
        n+\abs[a_0]+\abs[a_1]+\cdots+\abs[a_n]=N.
    \end{equation*}
    }
\end{enumerate}

