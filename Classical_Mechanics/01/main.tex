\chapter{Survey of the Elementary Principles}

The motion of material bodies formed the subject of some of the earliest research pursued by the pioneers of physics. From their efforts there has evolved a vast field known as analytical mechanics or dynamics, or simply, mechanics. In the present century the term ``classical mechanics" has come into wide use to denote this branch of physics in contradistinction to the newer physical theories, especially quantum mechanics. We shall follow this usage, interpreting the name to include the type of mechanics arising out of the special theory of relativity. It is the purpose of this book to develop the structure of classical mechanics and to outline some of its applications of present-day interest in pure physics. Basic to any presentation of mechanics are a number of fundamental physical concepts, such as space, time, simultaneity, mass, and force. For the most part, however, these concepts will not be analyzed critically here; rather, they will be assumed as undefined terms whose meanings are familiar to the reader.

\section{Definition and Existence of the Integral}

\begin{definition}\label{definition:6.1}
    Let \(\left[a,b\right]\) be a given interval. By a \emph{partition} \(P\) of \(\left[a,b\right]\) we mean a finite set of points \(x_0\), \(x_1\), \ldots, \(x_n\), where
    \begin{equation*}
        a=x_0\leqslant x_1\leqslant\cdots\leqslant x_{n-1}\leqslant x_n=b.
    \end{equation*}
    We write
    \begin{equation*}
        \increment x_i=x_i-x_{i-1}\qquad\text{(\(i=1,\ldots,n\))}.
    \end{equation*}
    Now suppose \(f\) is a bounded real function defined on \(\left[a,b\right]\). Corresponding to each partition \(P\) of \(\left[a,b\right]\) we put
    \begin{align*}
        M_i&=\sup f\left(x\right)\qquad\text{(\(x_{i-1}\leqslant x\leqslant x_i\))},\\
        m_i&=\inf f\left(x\right)\qquad\text{(\(x_{i-1}\leqslant x\leqslant x_i\))},\\
        U\left(P,f\right)&=\sum_{i=1}^nM_i\increment x_i,\\
        L\left(P,f\right)&=\sum_{i=1}^nm_i\increment x_i,
    \end{align*}
    and finally
    \begin{align}
        \upint_a^bf\odif{x}=\inf U\left(P,f\right),\label{eq:6.1}\\
        \lowint_a^bf\odif{x}=\sup L\left(P,f\right),\label{eq:6.2}
    \end{align}
    where the \(\inf\) and the \(\sup\) are taken over all partitions \(P\) of \(\left[a,b\right]\). The left members of \eqref{eq:6.1} and \eqref{eq:6.2} are called the \emph{upper} and \emph{lower Riemann integrals} of \(f\) over \(\left[a,b\right]\), respectively.

    If the upper and lower integrals are equal, we say that \(f\) is \emph{Riemann-integrable} on \(\left[a,b\right]\), we write \(f\in\symscr{R}\) (that is, \(\symscr{R}\) denotes the set of Riemann-integrable functions), and we denote the common value of \eqref{eq:6.1} and \eqref{eq:6.2} by
    \begin{equation}
        \int_a^bf\odif{x}
    \end{equation}
    or by
    \begin{equation}
        \int_a^bf\left(x\right)\odif{x}.
    \end{equation}

    This is the \emph{Riemann integral} of \(f\) over \(\left[a,b\right]\). Since \(f\) is bounded, there exist two numbers, \(m\) and \(M\), such that
    \begin{equation*}
        m\leqslant f\left(x\right)\leqslant M\qquad\text{(\(a\leqslant x\leqslant b\))}.
    \end{equation*}
    Hence, for every \(P\)
    \begin{equation*}
        m\left(b-a\right)\leqslant L\left(P,f\right)\leqslant U\left(P,f\right)\leqslant M\left(b-a\right),
    \end{equation*}
    so that the numbers \(L\left(P,f\right)\) and \(U\left(P,f\right)\) form a bounded set. This shows that \emph{the upper and lower integrals are defined for every} bounded function \(f\). The question of their equality, and hence the question of the integrability of \(f\), is a more delicate one. Instead ofinvestigating it separately for the Riemann integral, we shall immediately consider a more general situation.
\end{definition}

\begin{definition}
    Let \(\alpha\) be a monotonically increasing function on \(\left[a,b\right]\) (since \(\alpha\left(a\right)\) and \(\alpha\left(b\right)\) are finite, it follows that \(\alpha\) is bounded on \(\left[a,b\right]\)). Corresponding to each partition \(P\) of \(\left[a,b\right]\), we write
    \begin{equation*}
        \increment\alpha_i=\alpha\left(x_i\right)-\alpha\left(x_{i-1}\right).
    \end{equation*}
    It is clear that \(\increment\alpha_i\geqslant0\). For any real function \(f\) which is bounded on \(\left[a,b\right]\) we put
    \begin{align*}
        U\left(P,f,\alpha\right)=\sum_{i=1}^nM_i\increment\alpha_i,\\
        L\left(P,f,\alpha\right)=\sum_{i=1}^nm_i\increment\alpha_i,
    \end{align*}
    where \(M_i\), \(m_i\) have the same meaning as in Definition~\ref{definition:6.1}, and we define
    \begin{align}
        \upint_a^bf\odif{\alpha}=\inf U\left(P,f,\alpha\right),\label{eq:6.5}\\
        \lowint_a^bf\odif{\alpha}=\sup L\left(P,f,\alpha\right),\label{eq:6.6}
    \end{align}
    the \(\inf\) and \(\sup\) again being taken over all partitions.

    If the left members of \eqref{eq:6.5} and \eqref{eq:6.6} are equal, we denote their common value by
    \begin{equation}
        \int_a^bf\odif{\alpha}\label{eq:6.7}
    \end{equation}
    or sometimes by
    \begin{equation}
        \int_a^bf\left(x\right)\odif{\alpha\left(x\right)}.\label{eq:6.8}
    \end{equation}

    This is the \emph{Riemann--Stieltjes integral} (or simply the \emph{Stieltjes integral}) of \(f\) with respect to \(\alpha\), over \(\left[a,b\right]\).

    If \eqref{eq:6.7} exists, i.e., if \eqref{eq:6.5} and \eqref{eq:6.6} are equal, we say that \(f\) is integrable with respect to \(\alpha\), in the Riemann sense, and write \(f\in\symscr{R}\left(\alpha\right)\).
\end{definition}

By taking \(\alpha\left(x\right)=x\), the Riemann integral is seen to be a special case of the Riemann--Stieltjes integral. Let us mention explicitly, however, that in the general case \(\alpha\) need not even be continuous.

A few words should be said about the notation. We prefer \eqref{eq:6.7} to \eqref{eq:6.8}, since the letter \(x\) which appears in \eqref{eq:6.8} adds nothing to the content of \eqref{eq:6.7}. It is immaterial which letter we use to represent the so-called ``variable of integration." For instance, \eqref{eq:6.8} is the same as
\begin{equation*}
    \int_a^bf\left(y\right)\odif{\alpha\left(y\right)}.
\end{equation*}
The integral depends on \(f\), \(\alpha\), \(a\) and \(b\), but not on the variable of integration, which may as well be omitted.

The role played by the variable of integration is quite analogous to that of the index of summation: The two symbols
\begin{equation*}
    \sum_{i=1}^nc_i,\qquad\sum_{k=1}^nc_k
\end{equation*}
are the same, since each means \(c_1+c_2+\cdots+c_n\).

Of course, no harm is done by inserting the variable of integration, and in many cases it is actually convenient to do so.

We shall now investigate the existence of the integral \eqref{eq:6.7}. Without saying so every time, \(f\) will be assumed real and bounded, and \(\alpha\) monotonically increasing on \(\left[a,b\right]\); and, when there can be no misunderstanding, we shall write \(\int\) in place of \(\int_a^b\).

\begin{definition}
    We say that the partition \(P^*\) is a \emph{refinement} of \(P\) if \(P^*\supset P\) (that is, if every point of \(P\) is a point of \(P^*\)). Given two partitions, \(P_1\) and \(P_2\), we say that \(P^*\) is their \emph{common refinement} if \(P^*=P_1\cup P_2\).
\end{definition}

\begin{theorem}\label{theorem:6.1}
    If \(P^*\) is a refinement of \(P\), then
    \begin{equation}
        L\left(P,f,\alpha\right)\leqslant L\left(P^*,f,\alpha\right)\label{eq:6.9}
    \end{equation}
    and
    \begin{equation}
        U\left(P^*,f,\alpha\right)\leqslant U\left(P,f,\alpha\right).\label{eq:6.10}
    \end{equation}
\end{theorem}

\begin{proof}
    To prove \eqref{eq:6.9}, suppose first that \(P^*\) contains just one point more than \(P\). Let this extra point be \(x^*\), and suppose \(x_{i-1}<x^*<x_i\), where \(x_{i-1}\) and \(x_i\) are two consecutive points of \(P\). Put
    \begin{align*}
        w_1&=\inf f\left(x\right)\qquad\text{(\(x_{i-1}\leqslant x\leqslant x^*\))},\\
        w_2&=\inf f\left(x\right)\qquad\text{(\(x^*\leqslant x\leqslant x_i\))}.
    \end{align*}
    Clearly \(w_1>m_i\) and \(w_2>m_i\), where, as before,
    \begin{equation*}
        m_i=\inf f\left(x\right)\qquad\text{(\(x_{i-1}\leqslant x\leqslant x_i\))}.
    \end{equation*}
    Hence
    \begin{align*}
        &\phantom{{}={}}L\left(P^*,f,\alpha\right)-L\left(P,f,\alpha\right)\\
        &=w_1\left[\alpha\left(x^*\right)-\alpha\left(x_{i-1}\right)\right]+w_2\left[\alpha\left(x_i\right)-\alpha\left(x^*\right)\right]-m_i\left[\alpha\left(x_i\right)-\alpha\left(x_{i-1}\right)\right]\\
        &=\left(w_1-m_i\right)\left[\alpha\left(x^*\right)-\alpha\left(x_{i-1}\right)\right]+\left(w_2-m_i\right)\left[\alpha\left(x_i\right)-\alpha\left(x^*\right)\right]\geqslant0.
    \end{align*}
    If \(P^*\) contains \(k\) points more than \(P\), we repeat this reasoning \(k\) times, and arrive at \eqref{eq:6.9}. The proof of \eqref{eq:6.10} is analogous.
\end{proof}

\begin{theorem}
    \(\lowint_a^bf\odif{\alpha}\leqslant\upint_a^bf\odif{\alpha}\).
\end{theorem}

\begin{proof}
    Let \(P^*\) be the common refinement of two partitions \(P_1\), and \(P_2\). By Theorem~\ref{theorem:6.1},
    \begin{equation*}
        L\left(P_1,f,\alpha\right)\leqslant L\left(P^*,f,\alpha\right)\leqslant U\left(P^*,f,\alpha\right)\leqslant U\left(P_2,f,\alpha\right)
    \end{equation*}
    Hence
    \begin{equation}
        L\left(P_1,f,\alpha\right)\leqslant U\left(P_2,f,\alpha\right).\label{eq:6.11}
    \end{equation}
    If \(P_2\) is fixed and the \(\sup\) is taken over all \(P_1\), \eqref{eq:6.11} gives
    \begin{equation}
        \lowint f\odif{\alpha}\leqslant U\left(P_2,f,\alpha\right).\label{eq:6.12}
    \end{equation}
    The theorem follows by taking the \(\inf\) over all \(P_2\) in \eqref{eq:6.12}.
\end{proof}

\begin{theorem}\label{theorem:6.3}
    \(f\in\symscr{R}\left(a\right)\) on \(\left[a,b\right]\) if and only if for every \(\varepsilon>0\) there exists a partition \(P\) such that
    \begin{equation}
        U\left(P,f,\alpha\right)-L\left(P,f,\alpha\right)<\varepsilon.\label{eq:6.13}
    \end{equation}
\end{theorem}

\begin{proof}
    For every \(P\) we have
    \begin{equation*}
        L\left(P,f,\alpha\right)\leqslant\lowint f\odif{\alpha}\leqslant\upint f\odif{\alpha}\leqslant U\left(P,f,\alpha\right).
    \end{equation*}
    Thus \eqref{eq:6.13} implies
    \begin{equation*}
        0\leqslant\upint f\odif{\alpha}-\lowint f\odif{\alpha}<\varepsilon.
    \end{equation*}
    Hence, if \eqref{eq:6.13} can be satisfied for every \(\varepsilon>0\), we have
    \begin{equation*}
        \upint f\odif{\alpha}=\lowint f\odif{\alpha},
    \end{equation*}
    that is, \(f\in\symscr{R}\left(\alpha\right)\).

    Conversely, suppose \(f\in\symscr{R}\left(\alpha\right)\), and let \(\varepsilon>0\) be given. Then there exist partitions \(P_1\) and \(P_2\) such that
    \begin{align}
        U\left(P_2,f,\alpha\right)-\int f\odif{\alpha}&<\frac{\varepsilon}{2},\label{eq:6.14}\\
        \int f\odif{\alpha}-L\left(P_1,f,\alpha\right)&<\frac{\varepsilon}{2}.\label{eq:6.15}
    \end{align}
    We choose \(P\) to be the common refinement of \(P_1\) and \(P_2\). Then Theorem~\ref{theorem:6.1}, together with \eqref{eq:6.14} and \eqref{eq:6.15}, shows that
    \begin{equation*}
        U\left(P,f,\alpha\right)\leqslant U\left(P_2,f,\alpha\right)<\int f\odif{\alpha}+\frac{\varepsilon}{2}<L\left(P_1,f,\alpha\right)+\varepsilon\leqslant L\left(P,f,\alpha\right)+\varepsilon,
    \end{equation*}
    so that \eqref{eq:6.13} holds for this partition \(P\).
\end{proof}

Theorem~\ref{theorem:6.3} furnishes a convenient criterion for integrability. Before we apply it, we state some closely related facts.

% 6.7 Theorem (a) IfXI3) holdsfor some P and some e, then (13) holds (with the same e) for every refinement ofP. (b) If (13) holds for P = {x , . . . ,  xn}  and if st , /, are arbitrary points in il/W-/(',)|Aa,<e. 1=1 (c) //ye £?(a) and the hypotheses of(b) hold, then < I J=l JA Proof Theorem 6.4 implies (a). Under the assumptions made in (b), bothf(Si) andf(t{ ) lie in [w, ,  A/,], so that \f(s{)-/(/,) |  < vV/,- w, .  Thus I l/(*i)-/Ml Aa, * W/> «) ~ W>/> «), i=l which proves (b). The obvious inequalities £*/>,/, a) < Zfttd Aa, < U(P,fx) and prove (c). L(P,/, a) < Ifda < U(P,f a) 6.8 Theorem 7/"/w continuous on [a, b] thenfe £?(a) on [a, b]. (16) Proof Let e > be given. Choose n > so that [«(&)- <x(a)]n < e. Since /is uniformly continuous on [a, b] (Theorem 4.19), there exists a 6 > such that \f(x)-f(t)\ <rj = 126 PRINCIPLES OF MATHEMATICAL ANALYSIS (17) if x e [a, b), t 6 [a, b], and  |  x- t \  < S. If jP is any partition of [a, b] such that A.v, < 6 for all /,  then (16) implies that Mi-m^t] (f-1, ...,#») and therefore {/(/>,/, a)- L(P,/ a) = Y (Mi- mj Aa, n < >7 = i » £ Aa,- = //[a(&)- a(a)] < e. By Theorem 6.6, /e #(a). 6.9 Theorem Iff is monotonic on [a, b], and if a is continuous on [a, b], then fe ^2(a). (We still assume, ofcourse, that a is monotonic.) Proof Let e > be given. For any positive integer n, choose a partition such that Aa, (i n = \, ...,n). This is possible since a is continuous (Theorem 4.23). We suppose that/is monotonically increasing (the proof is analogous in the other case). Then M^Ax,), so that /»,=/(*,_,) (i = 1,...,/i), U(P,f, a)- UPJ, «) = g(6) ~ M(a) f [/(*,)-/(or,.,)] n if a? is taken large enough. By Theorem 6.6, fe 3?(ct). 6.10 Theorem Suppose f is bounded on [a, b], f has only finitely many points of discontinuity on [a, b], and a is continuous at every point at which f is discon tinuous. Thenfe@(cc). Proof Let e > be given. Put M = sup \f(x)\ ,  let  E be the set of points at which /is discontinuous. Since E is finite and a is continuous at every point of E, we can cover E by finitely many disjoint intervals [uj, Vj]  c [a, b] such that the sum of the corresponding differences a(Vj) — ol(Uj) is less than e. Furthermore, we can place these intervals in such a way that every point of E n (a, b) lies in the interior of some [uj ,  pj. THE RIEMANN-STIELTJES INTEGRAL 127 Remove the segments (w, ,  Vj) from [a, b). The remaining set K is compact. Hence /is uniformly continuous on K, and there exists 5 > such that |/(j)-/(/)| <£ \tseK, t e K, \s- t\  < S. Now form a partition P = {x ,  xlt . .  ., xn}  of [a, b], as follows: Each Uj occurs in P. Each Vj occurs in P. No point ofany segment (ujy vj) occurs in P. If a / _ 1  is not one of the Uj ,  then Ax, < 6. Note that A/,- m, < 2A/ for every /,  and that /V/ f  — m %  < e unless *,_! is one of the w,-. Hence, as in the proof of Theorem 6.8, U(P,f a)- L(?,/, a) < [a(Z>)- x(a)]e + 2Me. Since e is arbitrary, Theorem 6.6 shows that/e M{v). Note: Iffand a have a common point of discontinuity, then /need not be in &(y). Exercise 3 shows this. 6.11 Theorem Suppose fe Mix) on [a, b],  m <f< M, (j> is continuous on [m, M), and h(x) = (f>(f(x)) on [a, b]. Then h e #{?) on [a, b]. Proof Choose e > 0. Since $ is uniformly continuous on [m, A/], there exists (5>0 such that 6 < c and \<l>(s)- </>(t)\  <e if \s- t\  <5 and s, t e [w, M]. Sincefe %{?), there is a partition P = {x ,  x^, that (18) U(P,f, a)- L(P,f a) < S2 . xn}  of [a, b] such Let A/,-, nij have the same meaning as in Definition 6.1, and let A/,*, m* be the analogous numbers for h. Divide the numbers 1, .  .., n into two classes: / e A if A/,  — w, < S, i e B if  Mt  — m{  > 3. For ie'A, our choice of S shows that M* — mf < e. For i e 5, Mf- w* < 2K, where A"= sup| <£(0| ,  /" < f  < A/. By (18), we have (19) S X A«i < X (Af|- >»«) Aa,- < <5 2 <eB icfl so that Y,ieB Act-i  < £• It follows that U(P, h, a)- L(/>, /;, a) = X (M*- '«*) Aa,- + X M* ~ w?) Aa . i'e/4 / e B < e[a(6)- «(<?)] + 2Ac5 < e[a(ft)- a(fl) + 2A]. Since e was arbitrary, Theorem 6.6 implies that //  e @(cc). Remark: This theorem suggests the question: Just what functions are Riemann-integrable? The answer is given by Theorem 1  1.33(6).

\section{Characteristic Values}

The introductory remarks of the previous section provide us with a starting point for our attempt to analyze the general linear operator \(T\). We take our cue from \eqref{eq:6.2}, which suggests that we should study vectors which are sent by \(T\) into scalar multiples of themselves.

\begin{definition}
    Let \(V\) be a vector space over the field \(F\) and let \(T\) be a linear operator on \(V\). A \emph{characteristic value} of \(T\) is a scalar \(c\) in \(F\) such that there is a non-zero vector \(\alpha\) in \(V\) with \(T\alpha=c\alpha\). If \(c\) is a characteristic value of \(T\), then
    \begin{enumerate}
        \item any \(\alpha\) such that \(T\alpha=c\alpha\) is called a \emph{characteristic vector} of \(T\) associated with the characteristic value \(c\);
        \item the collection of all \(\alpha\) such that \(T\alpha=c\alpha\) is called the \emph{characteristic space} associated with \(c\).
    \end{enumerate}
\end{definition}

Characteristic values are often called characteristic roots, latent roots, eigenvalues, proper values, or spectral values. In this book we shall use only the name `characteristic values.'

If \(T\) is any linear operator and \(c\) is any scalar, the set of vectors \(\alpha\) such that \(T\alpha=c\alpha\) is a subspace of \(V\). It is the null space of the linear transformation \(\left(T-cI\right)\). We call \(c\) a characteristic value of \(T\) if this subspace is different from the zero subspace, i.e., if \(\left(T-cI\right)\) fails to be \(1\):\(1\). If the underlying space \(V\) is finite-dimensional, \(\left(T-cI\right)\) fails to be \(1\):\(1\) precisely when its determinant is different from \(0\). Let us summarize.

\begin{theorem}
    Let \(T\) be a linear operator on a finite-dimensional space \(V\) and let \(c\) he a scalar. The following are equivalent.
    \begin{enumerate}
        \item \(c\) is a characteristic value of \(T\).
        \item The operator \(\left(T-cI\right)\) is singular (not invertible).
        \item\label{itm:6.5} \(\det\left(T-cI\right)=0\).
    \end{enumerate}
\end{theorem}

The determinant criterion \ref{itm:6.5} is very important because it tells us where to look for the characteristic values of \(T\). Since \(\det\left(T-cI\right)\) is a polynomial of degree \(n\) in the variable \(c\), we will find the characteristic values as the roots of that polynomial. Let us explain carefully.

If \(\symcal{B}\) is any ordered basis for \(V\) and \(A=\left[T\right]_{\symcal{B}}\), then \(\left(T-cI\right)\) is invertible if and only if the matrix \(\left(A-cI\right)\) is invertible. Accordingly, we make the following definition.

\begin{definition}
    If \(A\) is an \(n\times n\) matrix over the field \(F\), a \emph{characteristic value of \(A\) in \(F\)} is a scalar \(c\) in \(F\) such that the matrix \(\left(A-cI\right)\) is singular (not invertible).
\end{definition}

Since \(c\) is a characteristic value of \(A\) if and only if \(\det\left(A-cI\right)=0\), or equivalently if and only if \(\det\left(cI-A\right)=0\), we form the matrix \(\left(xI-A\right)\) with polynomial entries, and consider the polynomial \(f=\det\left(xI-A\right)\). Clearly the characteristic values of \(A\) in \(F\) are just the scalars \(c\) in \(F\) such that \(f\left(c\right)=0\). For this reason \(f\) is called the \emph{characteristic polynomial} of \(A\). It is important to note that \(f\) is a monic polynomial which has degree exactly \(n\). This is easily seen from the formula for the determinant of a matrix in terms of its entries.

\begin{lemma}
    Similar matrices have the same characteristic polynomial.
\end{lemma}

\begin{proof}
    It \(B=P^{-1}AP\), then
    \begin{align*}
        \det\left(xI-B\right)&=\det\left(xI-P^{-1}AP\right)\\
        &=\det\left(P^{-1}\left(xI-A\right)P\right)\\
        &=\det P^{-1}\cdot\det\left(xI-A\right)\cdot\det P\\
        &=\det\left(xI-A\right).\qedhere
    \end{align*}
\end{proof}

This lemma enables us to define sensibly the characteristic polynomial of the operator \(T\) as the characteristic polynomial of any \(n\times n\) matrix which represents \(T\) in some ordered basis for \(V\). Just as for matrices, the characteristic values of \(T\) will be the roots of the characteristic polynomial for \(T\). In particular, this shows us that \(T\) cannot have more than \(n\) distinct characteristic values. It is important to point out that \(T\) may not have any characteristic values.

\begin{example}\label{example:6.1}
    Let \(T\) be the linear operator on \(R^2\) which is represented in the standard ordered basis by the matrix
    \begin{equation*}
        A=
        \begin{bmatrix}
            0 & -1 \\
            1 & 0
        \end{bmatrix}
        .
    \end{equation*}
    The characteristic polynomial for \(T\) (or for \(A\)) is
    \begin{equation*}
        \det\left(xI-A\right)=
        \begin{vmatrix}
            x & 1 \\
            -1 & x
        \end{vmatrix}
        =x^2+1.
    \end{equation*}
    Since this polynomial has no real roots, \(T\) has no characteristic values. If \(U\) is the linear operator on \(C^2\) which is represented by \(A\) in the standard ordered basis, then \(U\) has two characteristic values, \(\iu\) and \(-\iu\). Here we see a subtle point. In discussing the characteristic values of a matrix \(A\), we must be careful to stipulate the field involved. The matrix \(A\) above has no characteristic values in \(R\), but has the two characteristic values \(\iu\) and \(-\iu\) in \(C\).
\end{example}

\begin{example}\label{example:6.2}
    Let \(A\) be the (real) \(3\times3\) matrix
    \begin{equation*}
        \begin{bmatrix}
            3 & 1 & -1 \\
            2 & 2 & -1 \\
            2 & 2 & 0
        \end{bmatrix}
        .
    \end{equation*}
    Then the characteristic polynomial for \(A\) is
    \begin{equation*}
        \begin{vmatrix}
            x-3 & -1 & 1 \\
            -2 & x-2 & 1 \\
            -2 & -2 & x
        \end{vmatrix}
        =x^3-5x^2+8x-4=\left(x-1\right)\left(x-2\right)^2.
    \end{equation*}
    Thus the characteristic values of \(A\) are \(1\) and \(2\).
    
    Suppose that \(T\) is the linear operator on \(R^3\) which is represented by \(A\) in the standard basis. Let us find the characteristic vectors of \(T\) associated with the characteristic values, \(1\) and \(2\). Now
    \begin{equation*}
        A-I=
        \begin{bmatrix}
            2 & 1 & -1 \\
            2 & 1 & -1 \\
            2 & 2 & -1
        \end{bmatrix}
        .
    \end{equation*}
    It is obvious at a glance that \(A-I\) has rank equal to \(2\) (and hence \(T-I\) has nullity equal to \(1\)). So the space of characteristic vectors associated with the characteristic value \(1\) is one-dimensional. The vector \(\alpha_1=\left(1,0,2\right)\) spans the null space of \(T-I\). Thus \(T\alpha=\alpha\) if and only if \(\alpha\) is a scalar multiple of \(\alpha_1\). Now consider
    \begin{equation*}
        A-2I=
        \begin{bmatrix}
            1 & 1 & -1 \\
            2 & 0 & -1 \\
            2 & 2 & -2
        \end{bmatrix}
        .
    \end{equation*}
    Evidently \(A-2I\) also has rank \(2\), so that the space of characteristic vectors associated with the characteristic value \(2\) has dimension \(1\). Evidently \(T\alpha=2\alpha\) if and only if \(\alpha\) is a scalar multiple of \(\alpha_2=\left(1,1,2\right)\).
\end{example}

\begin{definition}
    Let \(T\) be a linear operator on the finite-dimensional space \(V\). We say that \(T\) is \emph{diagonalizable} if there is a basis for \(V\) each vector of which is a characteristic vector of \(T\).
\end{definition}

The reason for the name should be apparent; for, if there is an ordered basis \(\symcal{B}=\set{\alpha_1,\ldots,\alpha_n}\) for \(V\) in which each \(\alpha_i\) is a characteristic vector of \(T\), then the matrix of \(T\) in the ordered basis \(\symcal{B}\) is diagonal. If \(T_{\alpha_i}=c_i\alpha_i\), then
\begin{equation*}
    \left[T\right]_{\symcal{B}}=
    \begin{bmatrix}
        c_1 & 0 & \cdots & 0 \\
        0 & c_2 & \cdots & 0 \\
        \vdots & \vdots && \vdots \\
        0 & 0 & \cdots & c_n
    \end{bmatrix}
    .
\end{equation*}
We certainly do not require that the scalars \(c_1,\ldots,c_n\) be distinct; indeed, they may all be the same scalar (when \(T\) is a scalar multiple of the identity operator).

One could also define \(T\) to be diagonalizable when the characteristic vectors of \(T\) span \(V\). This is only superficially different from our definition, since we can select a basis out of any spanning set of vectors.

For Examples~\ref{example:6.1} and \ref{example:6.2} we purposely chose linear operators \(T\) on \(R^n\) which are not diagonalizable. In Example~\ref{example:6.1}, we have a linear operator on \(R^2\) which is not diagonalizable, because it has no characteristic values. In Example~\ref{example:6.2}, the operator \(T\) has characteristic values; in fact, the characteristic polynomial for \(T\) factors completely over the real number field: \(f=\left(x-1\right)\left(x-2\right)^2\). Nevertheless \(T\) fails to be diagonalizable. There is only a one-dimensional space of characteristic vectors associated with each of the two characteristic values of \(T\). Hence, we cannot possibly form a basis for \(R^3\) which consists of characteristic vectors of \(T\).

% Suppose that \(T\) is a diagonalizable linear operator. Let \(c_1\), \ldots, \(c_k\) be the \emph{distinct} characteristic values of \(T\). Then there is an ordered basis \(\symcal{B}\) in which \(T\) is represented by a diagonal matrix which has for its diagonal entries the scalars \(c_i\), each repeated a certain number of times. If c; is repeated di times, then (we may arrange that) the matrix has the block form 0 • • • (6-3) rrp-1 L-* Jffi = Ci/i 0 0 C2/2 0 * * * • • • 0 " 0 Cklk~ where Ij is the dj X dj identity matrix. From that matrix we see two things. First, the characteristic polynomial for T is the product of (possibly repeated) linear factors: 185 186 Elementary Canonical Forms f = (x — Ci)dl • • • (x — ck)dk. Chap. 6 If the scalar field F is algebraically closed, e.g., the field of complex num¬ bers, every polynomial over F can be so factored (see Section 4.5); however, if F is not algebraically closed, we are citing a special property of T when we say that its characteristic polynomial has such a factorization. The second thing we see from (6-3) is that dt, the number of times which d is repeated as root of /, is equal to the dimension of the space of characteristic vectors associated with the characteristic value c*. That is because the nullity of a diagonal matrix is equal to the number of zeros which it has on its main diagonal, and the matrix T — ct-7]® has di zeros on its main diagonal. This relation between the dimension of the characteristic space and the multiplicity of the characteristic value as a root of / does not seem exciting at first; however, it will provide us with a simpler way of deter¬ mining whether a given operator is diagonalizable. Lemma. Suppose that Ta = ca. If f is any polynomial, then f(T)a = i(c)a. Proof. Exercise. Lemma. Let T be a linear operator on the finite-dimensional space V. Let Ci, . . . , Ck be the distinct characteristic values of T and let Wi be the space of characteristic vectors associated with the characteristic value Ci. If W = Wi + • • • + Wk, then dim W = dim Wi + • • • + dim Wk. In fact, if ©i is an ordered basis for Wi, then © = (©i, . . . , ©k) is an ordered basis for W. Proof. The space W = Wi -- • • • + IE* is the subspace spanned by all of the characteristic vectors of T. Usually when one forms the sum W of subspaces Wi, one expects that dim W < dim Wi + • • • + dim Wk because of linear relations which may exist between vectors in the various spaces. This lemma states that the characteristic spaces associated with different characteristic values are independent of one another. Suppose that (for each i) we have a vector in Wi, and assume that /3i + * • * + /3* = 0. We shall show that (3i = 0 for each i. Let / be any polynomial. Since T/3i = cfti, the preceding lemma tells us that o = f(T)o = f(T)h + • • • +/(m = /(ci)/3i + • • • + f(ck)(3k. Choose polynomials f, ... ,fk such that fiifii) = in = -T t = J i  j- Sec. 6.2 Then Characteristic Values 0 = fi(T)0 = 2 Mi 3 = Now, let (Bi be an ordered basis for Wi, and let (B be the sequence (B = ((Bi, . . . , (Bk). Then (B spans the subspace W = Wifi- • • • + Wk. Also, (B is a linearly independent sequence of vectors, for the following reason. Any linear relation between the vectors in (B will have the form pi -f- • • • + Pk = 0, where Pi is some linear combination of the vectors in (Bi. From what we just did, we know that Pi = 0 for each i. Since each (B; is linearly independent, we see that we have only the trivial linear relation between the vectors in (B. Theorem 2. Let T be a linear operator on a finite-dimensional space V. Let Ci, . . . , Ck be the distinct characteristic values of T and let Wi be the null space of (T — cj). The following are equivalent. (i) T is diagonalizable. (ii) The characteristic polynomial for T is f = (X — Ci)dl • • * (X — Ck)dk and dim Wi = di, i = 1, . . . , k. (iii) dim Wi + • • • + dim Wk = dim V. Proof. We have observed that (i) implies (ii). If the characteristic polynomial / is the product of linear factors, as in (ii), then d + • • • + dk = dim V. For, the sum of the d/s is the degree of the characteristic polynomial, and that degree is dim V. Therefore (ii) implies (iii). Suppose (iii) holds. By the lemma, we must have V = Wifi- • • • + Wk, he., the characteristic vectors of T span V. The matrix analogue of Theorem 2 may be formulated as follows. Let A be an n X n matrix with entries in a field F, and let ci, . . . , c* be the distinct characteristic values of A in F. For each i, let Wi be the space of column matrices X (with entries in F) such that (A - cJ)X = 0, and let (B; be an ordered basis for Wi. The bases (Bi, . . . , (B* collectively string together to form the sequence of columns of a matrix P: P = Pj P2; • • • ] = (®i, . . . , (Bk). The matrix A is similar over F to a diagonal matrix if and only if P is a square matrix. When P is square, P is invertible and P~lAP is diagonal. Example 3. Let T be the linear operator on Rz which is represented in the standard ordered basis by the matrix 187 188 Elementary Canonical Forms Chap. 6 -6 -6 4 2 -6 -4 Let us indicate how one might compute the characteristic polynomial, using various row and column operations: x — 5 6 6 x — 5 0 6 1 -3 x — 4 -2  1 x - 2 -2 6 x + 4 -3 2 - x x + 4 = (* - 2) = (* - 2) x — 5 1 -3 x — 5 1 -2 = (* - 2) x — 5 0 1 6 -2 — 1 x + 4 0 1 6 -2 0 x + 2 6 x + 2 -2 = (x — 2)(x2 — 3x + 2) = (x — 2)2(x — 1). What are the dimensions of the spaces of characteristic vectors associated with the two characteristic values? We have 1 II 4 -6 -6 -1 CO CO i CO 2 -6 -5 -6 -6 A -21 = -1 2 CO 1 to -6 -6 We know that A — I is singular and obviously rank (A — I) > 2. There¬ fore, rank (A — I) = 2. It is evident that rank (A — 21) = 1. Let W2 be the spaces of characteristic vectors associated with the characteristic values 1, 2. We know that dim W = 1 and dim W2 = 2. By Theorem 2, T is diagonalizable. It is easy to exhibit a basis for R3 in which T is represented by a diagonal matrix. The null space of (T — I) is spanned by the vector ai = (3, — 1, 3) and so ai is a basis for W. The null space of T — 21 (i.e., the space W2) consists of the vectors (xi, x2, x3) with Xi = 2x2 -f 2x3. Thus, one example of a basis for W2 is = (2, 1, 0) *3 = (2, 0, 1). If (B = «i, a2, a3, then [T]® is the diagonal matrix Sec. 6.2 Characteristic Values D = “1 0 0 2 0 0 0 0 2 The fact that T is diagonalisable means that the original matrix A is similar (over R) to the diagonal matrix D. The matrix P which enables us to change coordinates from the basis (B to the standard basis is (of course) the matrix which has the transposes of P = a2, a3 as its column vectors: 2 2 1 0 0 1 Furthermore, AP = PD, so that P~lAP = D. Exercises 1. In each of the following cases, let T be the linear operator on R2 which is represented by the matrix A in the standard ordered basis for R2, and let U be the linear operator on C2 represented by A in the standard ordered basis. Find the characteristic polynomial for T and that for U, find the characteristic values of each operator, and for each such characteristic value c find a basis for the cor¬ responding space of characteristic vectors. 2. Let V be an n-dimensional vector space over F. What is the characteristic polynomial of the identity operator on V? What is the characteristic polynomial for the zero operator? 3. Let A be an n X n triangular matrix over the field F. Prove that the charac¬ teristic values of A are the diagonal entries of A, i.e., the scalars An. 4. Let T be the linear operator on Rz which is represented in the standard ordered basis by the matrix 1 00 r -9 4 4“ 4  —16 CO 00 7 Prove that T is diagonalizable by exhibiting a basis for Rs, each vector of which is a characteristic vector of T. 5. Let Is A similar over the field R to a diagonal matrix? Is A similar over the field C to a diagonal matrix? 189 190 Elementary Canonical Forms Chap. 6 6. Let T be the linear operator on R4 which is represented in the standard ordered basis by the matrix 0 a 0 0 b 0 0 0 0 c O' 0 0 0 0 0 Under what conditions on a, b, and c is T diagonalizable? 7. Let T be a linear operator on the n-dimensional vector space V, and suppose that T has n distinct characteristic values. Prove that T is diagonalizable. 8. Let A and B be n X n matrices over the field F. Prove that if (I — AB) is invertible, then I — BA is invertible and (/ - BA)-1 = I + B(I - AB)-L4. 9. Use the result of Exercise 8 to prove that, if A and B are n X n matrices over the field F, then AB and BA have precisely the same characteristic values in F. 10. Suppose that A is a 2 X 2 matrix with real entries which is symmetric (A* = 4). Prove that A is similar over R to a diagonal matrix. 11. Let N be a 2 X 2 complex matrix such that N2 = 0. Prove that either N = 0 or N is similar over C to 12. Use the result of Exercise 11 to prove the following: If 4 is a 2 X 2 matrix with complex entries, then A is similar over C to a matrix of one of the two types 1.0 fa 01 b] i O e Li flJ 13. Let V be the vector space of all functions from R into R which are continuous, i.e., the space of continuous real-valued functions on the real line. Let T be the linear operator on V defined by (TfKx) = J0xmdt. Prove that T has no characteristic values. 14. Let A be an n X n diagonal matrix with characteristic polynomial (x — ci)* • • • (x — ck)dk, where Ci, . . . , c* are distinct. Let V be the space of n X n matrices B such that AB = BA. Prove that the dimension of V is d + • • • + d2k. 15. Let V be the space of n X n matrices over F. Let A be a fixed n X n matrix over F. Let T be the linear operator ‘left multiplication by A* on V. Is it true that A and T have the same characteristic values?

