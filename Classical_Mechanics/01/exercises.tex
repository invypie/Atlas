\section*{Exercises}

\begin{enumerate}[resume]
    \item Consider a uniform thin disk that rolls without slipping on a horizontal plane. A horizontal force is applied to the center of the disk and in a direction parallel to the plane of the disk.
    \begin{enumerate}
        \item Derive Lagrange's equations and find the generalized force.
        \item Discuss the motion if the force is not applied parallel to the plane of the disk.
    \end{enumerate}
    \item The \emph{escape velocity} of a particle on Earth is the minimum velocity required at Earth's surface in order that the particle can escape from Earth's gravitational field. Neglecting the resistance of the atmosphere, the system is conservative. From the conservation theorem for potential plus kinetic energy show that the escape velocity for Earth, ignoring the presence of the Moon, is \(\qty{11.2}{km/s}\).
    \item Rockets are propelled by the momentum reaction of the exhaust gases expelled from the tail. Since these gases arise from the reaction of the fuels carried in the rocket, the mass of the rocket is not constant, but decreases as the fuel is expended. Show that the equation of motion for a rocket projected vertically upward in a uniform gravitational field, neglecting atmospheric friction, is
    \begin{equation*}
        m\odv{v}{t}=-v'\odv{m}{t}-mg,
    \end{equation*}
    where \(m\) is the mass of the rocket and \(v'\) is the velocity of the escaping gases relative to the rocket. Integrate this equation to obtain \(v\) as a function of \(m\), assuming a constant time rate of loss of mass. Show, for a rocket starting initially from rest, with \(v'\) equal to \(\qty{2.1}{km/s}\) and a mass loss per second equal to \(\rfrac{1}{60}\)th of the initial mass, that in order to reach the escape velocity the ratio of the weight of the fuel to the weight of the empty rocket must be almost \(300\)!
    \item Two points of mass \(m\) are joined by a rigid weightless rod of length \(l\), the center of which is constrained to move on a circle of radius \(a\). Express the kinetic energy in generalized coordinates.
    \item A point particle moves in space under the influence of a force derivable from a generalized potential of the form
    \begin{equation*}
        U\left(\symbf{r},\symbf{v}\right)=V\left(r\right)+\symbf{\sigma}\cdot\symbf{L},
    \end{equation*}
    where \(\symbf{r}\) is the radius vector from a fixed point, \(\symbf{L}\) is the angular momentum about that point, and \(\symbf{\sigma}\) is a fixed vector in space.
    \begin{enumerate}
        \item Find the components of the force on the particle in both Cartesian and spherical polar coordinates, on the basis of Eq.~\eqref{eq:1.58}.
        \item Show that the components in the two coordinate systems are related to each other as in Eq.~\eqref{eq:1.49}.
        \item Obtain the equations of motion in spherical polar coordinates.
    \end{enumerate}
    \item A particle moves in a plane under the influence of a force, acting toward a center of force, whose magnitude is
    \begin{equation*}
        F=\frac{1}{r^2}\left(1-\frac{\dot{r}^2-2\ddot{r}r}{c^2}\right),
    \end{equation*}
    where \(r\) is the distance of the particle to the center of force. Find the generalized potential that will result in such a force, and from that the Lagrangian for the motion in a plane. (The expression for \(F\) represents the force between two charges in Weber's electrodynamics.)
    \item A nucleus, originally at rest, decays radioactively by emitting an electron of momentum \(\qty{1.73}{MeV/\mathnormal{c}}\), and at right angles to the direction of the electron a neutrino with momentum \(\qty{1.00}{MeV/\mathnormal{c}}\). (The \(\unit{MeV}\), million electron volt, is a unit of energy used in modern physics, equal to \(\qty{1.60\times10^{-13}}{J}\). Correspondingly, \(\unit{MeV/\mathnormal{c}}\) is a unit of linear momentum equal to \(\qty{5.34\times10^{-22}}{kg.m/s}\).) In what direction does the nucleus recoil? What is its momentum in \(\unit{MeV/\mathnormal{c}}\)? If the mass of the residual nucleus is \(\qty{3.90\times10^{-25}}{kg}\) what is its kinetic energy, in electron volts?
    \item A Lagrangian for a particular physical system can be written as
    \begin{equation*}
        L'=\frac{m}{2}\left(a\dot{x}^2+2b\dot{x}\dot{y}+c\dot{y}^2\right)-\frac{K}{2}\left(ax^2+2bxy+cy^2\right),
    \end{equation*}
    where \(a\), \(b\), and \(c\) are arbitrary constants but subject to the condition that \(b^2-ac\ne0\). What are the equations of motion? Examine particularly the two cases \(a=0=c\) and \(b=0\), \(c=-a\). What is the physical system described by the above Lagrangian? Show that the usual Lagrangian for this system as defined by Eq.~\eqref{eq:1.57'} is related to \(L'\) by a point transformation (cf. Derivation~\ref{derivation:1.10}). What is the significance of the condition on the value of \(b^2-ac\)?
    \item Obtain the Lagrange equations of motion for a spherical pendulum, i.e., a mass point suspended by a rigid weightless rod.
    \item\label{exercise:1.20} A particle of mass \(m\) moves in one dimension such that it has the Lagrangian
    \begin{equation*}
        L=\frac{m^2\dot{x}^4}{12}+m\dot{x}^2V\left(x\right)-V^2\left(x\right),
    \end{equation*}
    where \(V\) is some differentiable function of \(x\). Find the equation of motion for \(x\left(t\right)\) and describe the physical nature of the system on the basis of this equation.
    \item Two mass points of mass \(m_1\) and \(m_2\) are connected by a string passing through a hole in a smooth table so that \(m_1\) rests on the table surface and \(m_2\) hangs suspended. Assuming \(m_2\) moves only in a vertical line, what are the generalized coordinates for the system? Write the Lagrange equations for the system and, if possible, discuss the physical significance any of them might have. Reduce the problem to a single second-order differential equation and obtain a first integral of the equation. What is its physical significance? (Consider the motion only until \(m_1\) reaches the hole.)
    \item Obtain the Lagrangian and equations of motion for the double pendulum illustrated in Fig.~\ref{fig:1.4}, where the lengths of the pendula are \(l_1\) and \(l_2\) with corresponding masses \(m_1\) and \(m_2\).
    \item Obtain the equation of motion for a particle falling vertically under the influence of gravity when frictional forces obtainable from a dissipation function \(\frac{1}{2}kv^2\) are present. Integrate the equation to obtain the velocity as a function of time and show that the maximum possible velocity for a fall from rest is \(v=\rfrac{mg}{k}\).
    \item A spring of rest length \(L_a\) (no tension) is connected to a support at one end and has a mass \(M\) attached at the other. Neglect the mass of the spring, the dimension of the mass \(M\), and assume that the motion is confined to a vertical plane. Also, assume that the spring only stretches without bending but it can swing in the plane.
    \begin{enumerate}
        \item\label{itm:1.6.20} Using the angular displacement of the mass from the vertical and the length that the string has stretched from its rest length (hanging with the mass \(m\)), find Lagrange's equations.
        \item Solve these equations for small stretching and angular displacements.
        \item Solve the equations in part~\ref{itm:1.6.20} to the next order in both stretching and angular displacement. This part is amenable to hand calculations. Using some reasonable assumptions about the spring constant, the mass, and the rest length, discuss the motion. Is a resonance likely under the assumptions stated in the problem?
        \item (For analytic computer programs.) Consider the spring to have a total mass \(m\ll M\). Neglecting the bending of the spring, set up Lagrange's equations correctly to first order in \(m\) and the angular and linear displacements.
        \item (For numerical computer analysis.) Make sets of reasonable assumptions of the constants in part~\ref{itm:1.6.20} and make a single plot of the two coordinates as functions of time.
    \end{enumerate}
\end{enumerate}
