% !TEX root = ../main.tex

\startpart
  [
    title=Group Theory,
    reference=part:01,
  ]

  The modern treatment of abstract algebra begins with the disarmingly simple abstract definition of a \bold{group}. This simple definition quickly leads to difficult questions involving the structure of such objects. There are many specific examples of groups and the power of the abstract point of view becomes apparent when results for \emph{all} of these examples are obtained by proving a \emph{single} result for the abstract group.

  The notion of a group did not simply spring into existence, however, but is rather the culmination of a long period of mathematical investigation, the first formal definition of an abstract group in the form in which we use it appearing in 1882.\footnote{For most of the historical comments below, see the excellent book \emph{A History of Algebra}, by B. L. van der Waerden, Springer-Verlag, 1980 and the references there, particularly \emph{The Genesis of the Abstract Group Concept: A Contribution to the History of the Origin of Abstract Group Theory} (translated from the German by Abe Shenitzer), by H. Wussing, MIT Press, 1984. See also \emph{Number Theory, An Approach Through History from Hammurapai to Legendre}, by A. Weil, Birkhäuser, 1984.} The definition of an abstract group has its origins in extremely old problems in algebraic equations, number theory, and geometry, and arose because very similar techniques were found to be applicable in a variety of situations. As Otto Hölder (1859--1937) observed, one of the essential characteristics of mathematics is that after applying a certain algorithm or method of proof one then considers the scope and limits of the method. As a result, properties possessed by a number of interesting objects are frequently abstracted and the question raised: can one determine \emph{all} the objects possessing these properties? Attempting to answer such a question also frequently adds considerable understanding of the original objects under consideration. It is in this fashion that the definition of an abstract group evolved into what is, for us, the starting point of abstract algebra.

  We illustrate with a few of the disparate situations in which the ideas later formalized into the notion of an abstract group were used.

  \input part_01/chap_01/main

\stoppart
