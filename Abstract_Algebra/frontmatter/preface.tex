% !TEX root = ../main.tex

\startchapter
  [
    title=Preface to the Third Edition,
    list=Preface,
    bookmark=Preface,
    marking=Preface,
  ]

  The principal change from the second edition is the addition of Gröbner bases to this edition. The basic theory is introduced in a new Section~\in[sect:09.06]. Applications to solving systems of polynomial equations (elimination theory) appear at the end of this section, rounding it out as a self-contained foundation in the topic. Additional applications and examples are then woven into the treatment of affine algebraic sets and \m{k}-algebra homomorphisms in Chapter~\in[chap:15]. Although the theory in the latter chapter remains independent of Gröbner bases, the new applications, examples and computational techniques significantly enhance the development, and we recommend that Section~\in[sect:09.06] be read either as a segue to or in parallel with Chapter~\in[chap:15]. A wealth of exercises involving Gröbner bases, both computational and theoretical in nature, have been added in Section~\in[sect:09.06] and Chapter~\in[chap:15]. Preliminary exercises on Gröbner bases can (and should, as an aid to understanding the algorithms) be done by hand, but more extensive computations, and in particular most of the use of Gröbner bases in the exercises in Chapter~\in[chap:15], will likely require computer assisted computation.

  Other changes include a streamlining of the classification of simple groups of order \m{168} (Section~\in[sect:06.02]), with the addition of a uniqueness proof via the projective plane of order \m{2}. Some other proofs or portions of the text have been revised slightly. A number of new exercises have been added throughout the book, primarily at the ends of sections in order to preserve as much as possible the numbering schemes of earlier editions. In particular, exercises have been added on free modules over noncommutative rings (\in[sect:10.03]), on Krull dimension (\in[sect:15.03]), and on flat modules (\in[sect:10.05] and \in[sect:17.01]).

  As with previous editions, the text contains substantially more than can normally be covered in a one year course. A basic introductory (one year) course should probably include Part~\in[part:01] up through Section~\in[sect:05.03], Part~\in[part:02] through Section~\in[sect:09.05], Sections~\in[sect:10.01], \in[sect:10.02], \in[sect:10.03], \in[sect:11.01], \in[sect:11.02] and Part~\in[part:04]. Chapter~\in[chap:12] should also be covered, either before or after Part~\in[part:04]. Additional topics from Chapters~\in[chap:05], \in[chap:06], \in[chap:09], \in[chap:10] and \in[chap:11] may be interspersed in such a course, or covered at the end as time permits.

  Sections~\in[sect:10.04] and \in[sect:10.05] are at a slightly higher level of difficulty than the initial sections of Chapter~\in[chap:10], and can be deferred on a first reading for those following the text sequentially. The latter section on properties of exact sequences, although quite long, maintains coherence through a parallel treatment of three basic functors in respective subsections.

  Beyond the core material, the third edition provides significant flexibility for students and instructors wishing to pursue a number of important areas of modern algebra, either in the form of independent study or courses. For example, well integrated one semester courses for students with some prior algebra background might include the following: Section~\in[sect:09.06] and Chapters~\in[chap:15] and \in[chap:16]; or Chapters~\in[chap:10] and \in[chap:17]; or Chapters~\in[chap:05], \in[chap:06] and Part~\in[part:06]. Each of these would also provide a solid background for a follow-up course delving more deeply into one of many possible areas: algebraic number theory, algebraic topology, algebraic geometry, representation theory, Lie groups, etc.

  The choice of new material and the style for developing and integrating it into the text are in consonance with a basic theme in the book: the power and beauty that accrues from a rich interplay between different areas of mathematics. The emphasis throughout has been to motivate the introduction and development of important algebraic concepts using as many examples as possible. We have not attempted to be encyclopedic, but have tried to touch on many of the central themes in elementary algebra in a manner suggesting the very natural development of these ideas.

  A number of important ideas and results appear in the exercises. This is not because they are not significant, rather because they did not fit easily into the flow of the text but were too important to leave out entirely. Sequences of exercises on one topic are prefaced with some remarks and are structured so that they may be read without actually doing the exercises. In some instances, new material is introduced first in the exercises---often a few sections before it appears in the text---so that students may obtain an easier introduction to it by doing these exercises (e.g., Lagrange's Theorem appears in the exercises in Section~\in[sect:01.07] and in the text in Section~\in[sect:03.02]). All the exercises are within the scope of the text and hints are given [in brackets] where we felt they were needed. Exercises we felt might be less straightforward are usually phrased so as to provide the answer to the exercise; as well many exercises have been broken down into a sequence of more routine exercises in order to make them more accessible.

  We have also purposely minimized the functorial language in the text in order to keep the presentation as elementary as possible. We have refrained from providing specific references for additional reading when there are many fine choices readily available. Also, while we have endeavored to include as many fundamental topics as possible, we apologize if for reasons of space or personal taste we have neglected any of the reader's particular favorites.

  We are deeply grateful to and would like here to thank the many students and colleagues around the world who, over more than 15 years, have offered valuable comments, insights and encouragement---their continuing support and interest have motivated our writing of this third edition.

  \pagebreak[no]
  \blank

  \rightaligned{David Dummit\\ Richard Foote\\ June, 2003}

\stopchapter
