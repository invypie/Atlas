\section{Definition and Existence of the Integral}

\begin{definition}\label{definition:6.1}
    Let \(\left[a,b\right]\) be a given interval. By a \emph{partition} \(P\) of \(\left[a,b\right]\) we mean a finite set of points \(x_0\), \(x_1\), \(\ldots\), \(x_n\), where
    \begin{equation*}
        a=x_0\leqslant x_1\leqslant\cdots\leqslant x_{n-1}\leqslant x_n=b.
    \end{equation*}
    We write
    \begin{equation*}
        \increment x_i=x_i-x_{i-1}\qquad\text{(\(i=1,\ldots,n\))}.
    \end{equation*}
    Now suppose \(f\) is a bounded real function defined on \(\left[a,b\right]\). Corresponding to each partition \(P\) of \(\left[a,b\right]\) we put
    \begin{align*}
        M_i&=\sup f\left(x\right)\qquad\text{(\(x_{i-1}\leqslant x\leqslant x_i\))},\\
        m_i&=\inf f\left(x\right)\qquad\text{(\(x_{i-1}\leqslant x\leqslant x_i\))},\\
        U\left(P,f\right)&=\sum_{i=1}^nM_i\increment x_i,\\
        L\left(P,f\right)&=\sum_{i=1}^nm_i\increment x_i,
    \end{align*}
    and finally
    \begin{align}
        \upint_a^bf\odif{x}&=\inf U\left(P,f\right),\label{eq:6.1}\\
        \lowint_a^bf\odif{x}&=\sup L\left(P,f\right),\label{eq:6.2}
    \end{align}
    where the \(\inf\) and the \(\sup\) are taken over all partitions \(P\) of \(\left[a,b\right]\). The left members of \eqref{eq:6.1} and \eqref{eq:6.2} are called the \emph{upper} and \emph{lower Riemann integrals} of \(f\) over \(\left[a,b\right]\), respectively.

    If the upper and lower integrals are equal, we say that \(f\) is \emph{Riemann-integrable} on \(\left[a,b\right]\), we write \(f\in\symscr{R}\) (that is, \(\symscr{R}\) denotes the set of Riemann-integrable functions), and we denote the common value of \eqref{eq:6.1} and \eqref{eq:6.2} by
    \begin{equation}
        \int_a^bf\odif{x}
    \end{equation}
    or by
    \begin{equation}
        \int_a^bf\left(x\right)\odif{x}.
    \end{equation}

    This is the \emph{Riemann integral} of \(f\) over \(\left[a,b\right]\). Since \(f\) is bounded, there exist two numbers, \(m\) and \(M\), such that
    \begin{equation*}
        m\leqslant f\left(x\right)\leqslant M\qquad\text{(\(a\leqslant x\leqslant b\))}.
    \end{equation*}
    Hence, for every \(P\)
    \begin{equation*}
        m\left(b-a\right)\leqslant L\left(P,f\right)\leqslant U\left(P,f\right)\leqslant M\left(b-a\right),
    \end{equation*}
    so that the numbers \(L\left(P,f\right)\) and \(U\left(P,f\right)\) form a bounded set. This shows that \emph{the upper and lower integrals are defined for every} bounded function \(f\). The question of their equality, and hence the question of the integrability of \(f\), is a more delicate one. Instead of investigating it separately for the Riemann integral, we shall immediately consider a more general situation.
\end{definition}

\begin{definition}
    Let \(\alpha\) be a monotonically increasing function on \(\left[a,b\right]\) (since \(\alpha\left(a\right)\) and \(\alpha\left(b\right)\) are finite, it follows that \(\alpha\) is bounded on \(\left[a,b\right]\)). Corresponding to each partition \(P\) of \(\left[a,b\right]\), we write
    \begin{equation*}
        \increment\alpha_i=\alpha\left(x_i\right)-\alpha\left(x_{i-1}\right).
    \end{equation*}
    It is clear that \(\increment\alpha_i\geqslant0\). For any real function \(f\) which is bounded on \(\left[a,b\right]\) we put
    \begin{align*}
        U\left(P,f,\alpha\right)=\sum_{i=1}^nM_i\increment\alpha_i,\\
        L\left(P,f,\alpha\right)=\sum_{i=1}^nm_i\increment\alpha_i,
    \end{align*}
    where \(M_i\), \(m_i\) have the same meaning as in Definition~\ref{definition:6.1}, and we define
    \begin{align}
        \upint_a^bf\odif{\alpha}=\inf U\left(P,f,\alpha\right),\label{eq:6.5}\\
        \lowint_a^bf\odif{\alpha}=\sup L\left(P,f,\alpha\right),\label{eq:6.6}
    \end{align}
    the \(\inf\) and \(\sup\) again being taken over all partitions.

    If the left members of \eqref{eq:6.5} and \eqref{eq:6.6} are equal, we denote their common value by
    \begin{equation}
        \int_a^bf\odif{\alpha}\label{eq:6.7}
    \end{equation}
    or sometimes by
    \begin{equation}
        \int_a^bf\left(x\right)\odif{\alpha\left(x\right)}.\label{eq:6.8}
    \end{equation}

    This is the \emph{Riemann--Stieltjes integral} (or simply the \emph{Stieltjes integral}) of \(f\) with respect to \(\alpha\), over \(\left[a,b\right]\).

    If \eqref{eq:6.7} exists, i.e., if \eqref{eq:6.5} and \eqref{eq:6.6} are equal, we say that \(f\) is integrable with respect to \(\alpha\), in the Riemann sense, and write \(f\in\symscr{R}\left(\alpha\right)\).
\end{definition}

By taking \(\alpha\left(x\right)=x\), the Riemann integral is seen to be a special case of the Riemann--Stieltjes integral. Let us mention explicitly, however, that in the general case \(\alpha\) need not even be continuous.

A few words should be said about the notation. We prefer \eqref{eq:6.7} to \eqref{eq:6.8}, since the letter \(x\) which appears in \eqref{eq:6.8} adds nothing to the content of \eqref{eq:6.7}. It is immaterial which letter we use to represent the so-called ``variable of integration." For instance, \eqref{eq:6.8} is the same as
\begin{equation*}
    \int_a^bf\left(y\right)\odif{\alpha\left(y\right)}.
\end{equation*}
The integral depends on \(f\), \(\alpha\), \(a\) and \(b\), but not on the variable of integration, which may as well be omitted.

The role played by the variable of integration is quite analogous to that of the index of summation: The two symbols
\begin{equation*}
    \sum_{i=1}^nc_i,\qquad\sum_{k=1}^nc_k
\end{equation*}
are the same, since each means \(c_1+c_2+\cdots+c_n\).

Of course, no harm is done by inserting the variable of integration, and in many cases it is actually convenient to do so.

We shall now investigate the existence of the integral \eqref{eq:6.7}. Without saying so every time, \(f\) will be assumed real and bounded, and \(\alpha\) monotonically increasing on \(\left[a,b\right]\); and, when there can be no misunderstanding, we shall write \(\int\) in place of \(\int_a^b\).

\begin{definition}
    We say that the partition \(P^*\) is a \emph{refinement} of \(P\) if \(P^*\supset P\) (that is, if every point of \(P\) is a point of \(P^*\)). Given two partitions, \(P_1\) and \(P_2\), we say that \(P^*\) is their \emph{common refinement} if \(P^*=P_1\cup P_2\).
\end{definition}

\begin{theorem}\label{theorem:6.1}
    If \(P^*\) is a refinement of \(P\), then
    \begin{equation}
        L\left(P,f,\alpha\right)\leqslant L\left(P^*,f,\alpha\right)\label{eq:6.9}
    \end{equation}
    and
    \begin{equation}
        U\left(P^*,f,\alpha\right)\leqslant U\left(P,f,\alpha\right).\label{eq:6.10}
    \end{equation}
\end{theorem}

\begin{proof}
    To prove \eqref{eq:6.9}, suppose first that \(P^*\) contains just one point more than \(P\). Let this extra point be \(x^*\), and suppose \(x_{i-1}<x^*<x_i\), where \(x_{i-1}\) and \(x_i\) are two consecutive points of \(P\). Put
    \begin{align*}
        w_1&=\inf f\left(x\right)\qquad\text{(\(x_{i-1}\leqslant x\leqslant x^*\))},\\
        w_2&=\inf f\left(x\right)\qquad\text{(\(x^*\leqslant x\leqslant x_i\))}.
    \end{align*}
    Clearly \(w_1>m_i\) and \(w_2>m_i\), where, as before,
    \begin{equation*}
        m_i=\inf f\left(x\right)\qquad\text{(\(x_{i-1}\leqslant x\leqslant x_i\))}.
    \end{equation*}
    Hence
    \begin{align*}
        &\phantom{{}={}}L\left(P^*,f,\alpha\right)-L\left(P,f,\alpha\right)\\
        &=w_1\left[\alpha\left(x^*\right)-\alpha\left(x_{i-1}\right)\right]+w_2\left[\alpha\left(x_i\right)-\alpha\left(x^*\right)\right]-m_i\left[\alpha\left(x_i\right)-\alpha\left(x_{i-1}\right)\right]\\
        &=\left(w_1-m_i\right)\left[\alpha\left(x^*\right)-\alpha\left(x_{i-1}\right)\right]+\left(w_2-m_i\right)\left[\alpha\left(x_i\right)-\alpha\left(x^*\right)\right]\geqslant0.
    \end{align*}
    If \(P^*\) contains \(k\) points more than \(P\), we repeat this reasoning \(k\) times, and arrive at \eqref{eq:6.9}. The proof of \eqref{eq:6.10} is analogous.
\end{proof}

\begin{theorem}
    \(\lowint_a^bf\odif{\alpha}\leqslant\upint_a^bf\odif{\alpha}\).
\end{theorem}

\begin{proof}
    Let \(P^*\) be the common refinement of two partitions \(P_1\), and \(P_2\). By Theorem~\ref{theorem:6.1},
    \begin{equation*}
        L\left(P_1,f,\alpha\right)\leqslant L\left(P^*,f,\alpha\right)\leqslant U\left(P^*,f,\alpha\right)\leqslant U\left(P_2,f,\alpha\right)
    \end{equation*}
    Hence
    \begin{equation}
        L\left(P_1,f,\alpha\right)\leqslant U\left(P_2,f,\alpha\right).\label{eq:6.11}
    \end{equation}
    If \(P_2\) is fixed and the \(\sup\) is taken over all \(P_1\), \eqref{eq:6.11} gives
    \begin{equation}
        \lowint f\odif{\alpha}\leqslant U\left(P_2,f,\alpha\right).\label{eq:6.12}
    \end{equation}
    The theorem follows by taking the \(\inf\) over all \(P_2\) in \eqref{eq:6.12}.
\end{proof}

\begin{theorem}\label{theorem:6.3}
    \(f\in\symscr{R}\left(a\right)\) on \(\left[a,b\right]\) if and only if for every \(\varepsilon>0\) there exists a partition \(P\) such that
    \begin{equation}
        U\left(P,f,\alpha\right)-L\left(P,f,\alpha\right)<\varepsilon.\label{eq:6.13}
    \end{equation}
\end{theorem}

\begin{proof}
    For every \(P\) we have
    \begin{equation*}
        L\left(P,f,\alpha\right)\leqslant\lowint f\odif{\alpha}\leqslant\upint f\odif{\alpha}\leqslant U\left(P,f,\alpha\right).
    \end{equation*}
    Thus \eqref{eq:6.13} implies
    \begin{equation*}
        0\leqslant\upint f\odif{\alpha}-\lowint f\odif{\alpha}<\varepsilon.
    \end{equation*}
    Hence, if \eqref{eq:6.13} can be satisfied for every \(\varepsilon>0\), we have
    \begin{equation*}
        \upint f\odif{\alpha}=\lowint f\odif{\alpha},
    \end{equation*}
    that is, \(f\in\symscr{R}\left(\alpha\right)\).

    Conversely, suppose \(f\in\symscr{R}\left(\alpha\right)\), and let \(\varepsilon>0\) be given. Then there exist partitions \(P_1\) and \(P_2\) such that
    \begin{align}
        U\left(P_2,f,\alpha\right)-\int f\odif{\alpha}&<\frac{\varepsilon}{2},\label{eq:6.14}\\
        \int f\odif{\alpha}-L\left(P_1,f,\alpha\right)&<\frac{\varepsilon}{2}.\label{eq:6.15}
    \end{align}
    We choose \(P\) to be the common refinement of \(P_1\) and \(P_2\). Then Theorem~\ref{theorem:6.1}, together with \eqref{eq:6.14} and \eqref{eq:6.15}, shows that
    \begin{equation*}
        U\left(P,f,\alpha\right)\leqslant U\left(P_2,f,\alpha\right)<\int f\odif{\alpha}+\frac{\varepsilon}{2}<L\left(P_1,f,\alpha\right)+\varepsilon\leqslant L\left(P,f,\alpha\right)+\varepsilon,
    \end{equation*}
    so that \eqref{eq:6.13} holds for this partition \(P\).
\end{proof}

Theorem~\ref{theorem:6.3} furnishes a convenient criterion for integrability. Before we apply it, we state some closely related facts.

\begin{theorem}
    \leavevmode
    \begin{enumerate}
        \item\label{itm:6.1.1} If \eqref{eq:6.13} holds for some \(P\) and some \(\varepsilon\), then \eqref{eq:6.13} holds (with the same \(\varepsilon\)) for every refinement of \(P\).
        \item\label{itm:6.1.2} If \eqref{eq:6.13} holds for \(P=\set{x_0,\ldots,x_n}\) and if \(s_i\), \(t_i\) are arbitrary points in \(\left[x_{i-1},x_i\right]\), then
        \begin{equation*}
            \sum_{i=1}^n\abs[f\left(s_i\right)-f\left(t_i\right)]\increment\alpha_i<\varepsilon.
        \end{equation*}
        \item\label{itm:6.1.3} If \(f\in\symscr{R}\left(\alpha\right)\) and the hypotheses of \ref{itm:6.1.2} hold, then
        \begin{equation*}
            \abs[\sum_{i=1}^nf\left(t_i\right)\increment\alpha_i-\int_a^bf\odif{\alpha}]<\varepsilon.
        \end{equation*}
    \end{enumerate}
\end{theorem}

\begin{proof}
    Theorem~\ref{theorem:6.1} implies \ref{itm:6.1.1}. Under the assumptions made in \ref{itm:6.1.2}, both \(f\left(s_i\right)\) and \(f\left(t_i\right)\) lie in \(\left[m_i,M_i\right]\), so that \(\abs[f\left(s_i\right)-f\left(t_i\right)]\leqslant M_i-m_i\). Thus
    \begin{equation*}
        \sum_{i=1}^n\abs[f\left(s_i\right)-f\left(t_i\right)]\increment\alpha_i\leqslant U\left(P,f,\alpha\right)-L\left(P,f,\alpha\right),
    \end{equation*}
    which proves \ref{itm:6.1.2}. The obvious inequalities
    \begin{equation*}
        L\left(P,f,\alpha\right)\leqslant\sum f\left(t_i\right)\increment\alpha_i\leqslant U\left(P,f,\alpha\right)
    \end{equation*}
    and
    \begin{equation*}
        L\left(P,f,\alpha\right)\leqslant\int f\odif{\alpha}\leqslant U\left(P,f,\alpha\right)
    \end{equation*}
    prove \ref{itm:6.1.3}.
\end{proof}

\begin{theorem}\label{theorem:6.5}
    If \(f\) is continuous on \(\left[a,b\right]\) then \(f\in\symscr{R}\left(\alpha\right)\) on \(\left[a,b\right]\).
\end{theorem}

\begin{proof}
    Let \(\varepsilon>0\) be given. Choose \(\eta>0\) so that
    \begin{equation*}
        \left[\alpha\left(b\right)-\alpha\left(a\right)\right]\eta<\varepsilon.
    \end{equation*}
    Since \(f\) is uniformly continuous on \(\left[a,b\right]\) (Theorem~\ref{theorem:4.?}), there exists a \(\delta>0\) such that
    \begin{equation}
        \abs[f\left(x\right)-f\left(t\right)]<\eta\label{eq:6.16}
    \end{equation}
    if \(x\in\left[a,b\right]\), \(t\in\left[a,b\right]\), and \(\abs[x-t]<\delta\).

    If \(P\) is any partition of \(\left[a,b\right]\) such that \(\increment x_i<\delta\) for all \(i\),  then \eqref{eq:6.16} implies that
    \begin{equation}
        M_i-m_i\leqslant\eta\qquad\text{(\(i=1,\ldots,n\))}
    \end{equation}
    and therefore
    \begin{align*}
        U\left(P,f,\alpha\right)-L\left(P,f,\alpha\right)&=\sum_{i=1}^n\left(M_i-m_i\right)\increment\alpha_i\\
        &\leqslant\eta\sum_{i=1}^n\increment\alpha_i=\eta\left[\alpha\left(b\right)-\alpha\left(a\right)\right]<\varepsilon.
    \end{align*}
    By Theorem~\ref{theorem:6.3}, \(f\in\symscr{R}\left(\alpha\right)\).
\end{proof}

\begin{theorem}
    If \(f\) is monotonic on \(\left[a,b\right]\), and if \(\alpha\) is continuous on \(\left[a,b\right]\), then \(f\in\symscr{R}\left(\alpha\right)\). (We still assume, of course, that \(\alpha\) is monotonic.)
\end{theorem}

\begin{proof}
    Let \(\varepsilon>0\) be given. For any positive integer \(n\), choose a partition such that
    \begin{equation*}
        \increment\alpha_i=\frac{\alpha\left(b\right)-\alpha\left(a\right)}{n}\qquad\text{(\(i=1,\ldots,n\))}.
    \end{equation*}
    This is possible since \(\alpha\) is continuous (Theorem~\ref{theorem:4.?}).

    We suppose that \(f\) is monotonically increasing (the proof is analogous in the other case). Then
    \begin{equation*}
        M_i=f\left(x_i\right),\qquad m_i=f\left(x_{i-1}\right)\qquad\text{(\(i=1,\ldots,n\))},
    \end{equation*}
    so that
    \begin{align*}
        U\left(P,f,\alpha\right)-L\left(P,f,\alpha\right)&=\frac{\alpha\left(b\right)-\alpha\left(a\right)}{n}\sum_{i=1}^n\left[f\left(x_i\right)-f\left(x_{i-1}\right)\right]\\
        &=\frac{\alpha\left(b\right)-\alpha\left(a\right)}{n}\cdot\left[f\left(b\right)-f\left(a\right)\right]<\varepsilon
    \end{align*}
    if \(n\) is taken large enough. By Theorem~\ref{theorem:6.3}, \(f\in\symscr{R}\left(\alpha\right)\).
\end{proof}

\begin{theorem}
    Suppose \(f\) is bounded on \(\left[a,b\right]\), \(f\) has only finitely many points of discontinuity on \(\left[a,b\right]\), and \(\alpha\) is continuous at every point at which \(f\) is discontinuous. Then \(F\in\symscr{R}\left(\alpha\right)\).
\end{theorem}

\begin{proof}
    Let \(\varepsilon>0\) be given. Put \(M=\sup\abs[f\left(x\right)]\), let \(E\) be the set of points at which \(f\) is discontinuous. Since \(E\) is finite and \(\alpha\) is continuous at every point of \(E\), we can cover \(E\) by finitely many disjoint intervals \(\left[u_j,v_j\right]\subset\left[a,b\right]\) such that the sum of the corresponding differences \(\alpha\left(v_j\right)-\alpha\left(u_j\right)\) is less than \(\varepsilon\). Furthermore, we can place these intervals in such a way that every point of \(E\cap\left(a,b\right)\) lies in the interior of some \(\left[u_j,v_j\right]\).

    Remove the segments \(\left(u_j,v_j\right)\) from \(\left[a,b\right]\). The remaining set \(K\) is compact. Hence \(f\) is uniformly continuous on \(K\), and there exists \(\delta>0\) such that \(\abs[f\left(s\right)-f\left(t\right)]<\varepsilon\) if \(s\in K\), \(t\in K\), \(\abs[s-t]<\delta\).

    Now form a partition \(P=\set{x_0,x_1,\ldots,x_n}\) of \(\left[a,b\right]\), as follows: Each \(u_j\) occurs in \(P\). Each \(v_j\) occurs in \(P\). No point of any segment \(\left(u_j,v_j\right)\) occurs in \(P\). If \(x_{i-1}\) is not one of the \(u_j\), then \(\increment x_i<\delta\).

    Note that \(M_i-m_i\leqslant2M\) for every \(i\), and that \(M_i-m_i\leqslant\varepsilon\) unless \(x_{i-1}\) is one of the \(u_j\). Hence, as in the proof of Theorem~\ref{theorem:6.5},
    \begin{equation*}
        U\left(P,f,\alpha\right)-L\left(P,f,\alpha\right)\leqslant\left[\alpha\left(b\right)-\alpha\left(a\right)\right]\varepsilon+2M\varepsilon.
    \end{equation*}
    Since \(\varepsilon\) is arbitrary, Theorem~\ref{theorem:6.3} shows that \(f\in\symscr{R}\left(\alpha\right)\).
\end{proof}

\begin{note*}
    If \(f\) and \(\alpha\) have a common point of discontinuity, then \(f\) need not be in \(\symscr{R}\left(\alpha\right)\). Exercise~\ref{exercise:6.3} shows this.
\end{note*}

\begin{theorem}
    Suppose \(f\in\symscr{R}\left(\alpha\right)\) on \(\left[a,b\right]\), \(m\leqslant f\leqslant M\), \(\phi\) is continuous on \(\left[m,M\right]\), and \(h\left(x\right)=\phi\left(f\left(x\right)\right)\) on \(\left[a,b\right]\). Then \(h\in\symscr{R}\left(\alpha\right)\) on \(\left[a,b\right]\).
\end{theorem}

\begin{proof}
    Choose \(\varepsilon>0\). Since \(\phi\) is uniformly continuous on \(\left[m,M\right]\), there exists \(\delta>0\) such that \(\delta<\varepsilon\) and \(\abs[\phi\left(s\right)-\phi\left(t\right)]<\varepsilon\) if \(\abs[s-t]\leqslant\delta\) and \(s,t\in\left[m,M\right]\).

    Since \(f\in\symscr{R}\left(\alpha\right)\), there is a partition \(P=\set{x_0,x_1,\ldots,x_n}\) of \(\left[a,b\right]\) such that
    \begin{equation}
        U\left(P,f,\alpha\right)-L\left(P,f,\alpha\right)<\delta^2.\label{eq:6.18}
    \end{equation}
    Let \(M_i\), \(m_i\) have the same meaning as in Definition~\ref{definition:6.1}, and let \(M_i^*\), \(m_i^*\) be the analogous numbers for \(h\). Divide the numbers \(1\), \(\ldots\), \(n\) into two classes: \(i\in A\) if \(M_i-m_i<\delta\), \(i\in B\) if \(M_i-m_i\geqslant\delta\).

    For \(i\in A\), our choice of \(\delta\) shows that \(M_i^*-m_i^*\leqslant\varepsilon\).
    
    For \(i\in B\), \(M_i^*-m_i^*\leqslant2K\), where \(K=\sup\abs[\phi\left(t\right)]\), \(m\leqslant t\leqslant M\). By \eqref{eq:6.18}, we have
    \begin{equation}
        \delta\sum_{i\in B}\increment\alpha_i\leqslant\sum_{i\in B}\left(M_i-m_i\right)\increment\alpha_i<\delta^2
    \end{equation}
    so that \(\sum_{i\in B}\increment\alpha_i<\delta\). It follows that
    \begin{align*}
        U\left(P,h,\alpha\right)-L\left(P,h,\alpha\right)&=\sum_{i\in A}\left(M_i^*-m_i^*\right)\increment\alpha_i+\sum_{i\in B}\left(M_i^*-m_i^*\right)\increment\alpha_i\\
        &\leqslant\varepsilon\left[\alpha\left(b\right)-\alpha\left(a\right)\right]+2K\delta<\varepsilon\left[\alpha\left(b\right)-\alpha\left(a\right)+2K\right].
    \end{align*}
    Since \(\varepsilon\) was arbitrary, Theorem~\ref{theorem:6.3} implies that \(h\in\symscr{R}\left(\alpha\right)\).
\end{proof}

\begin{remark*}
    This theorem suggests the question: Just what functions are Riemann-integrable? The answer is given by Theorem~\ref{theorem:11.?}\ref{itm:11.?}.
\end{remark*}
